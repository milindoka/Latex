\documentclass{beamer} 
\usetheme{Berlin}
\usepackage{showexpl} 
\usepackage{tikz}
\usepackage{chemfig}
\usepackage[utf8]{inputenc}
\usepackage{cancel}
\usepackage{circuitikz}
\usepackage[english]{babel}
\usepackage{blindtext}
\usepackage{multicol}
\usepackage{ragged2e}



\title{Derivatives}
\subtitle{Derivatives - Practice}
\author{Milind Oka}



\lstloadlanguages{[LaTeX]Tex} 
\lstset{% 
     basicstyle=\ttfamily\small, 
     commentstyle=\itshape\ttfamily\small, 
     showspaces=false, 
     showstringspaces=false, 
     breaklines=true, 
     breakautoindent=true, 
     captionpos=t 
} 

\begin{document} 

\frame {
		\titlepage
	}

%%%%%%%%%%%%%% Start Slides %%%%%%%%%%%%% 
\huge

%%%~~~~~~~~~~~~~~~~~~~~~~~~~~~~~~~~~~~~~~~~~
\frame
{
\textbf{Question :} 
$\begin{aligned}[t] 
y= \sqrt{x}  \text{\ , \  \ find \ \  } \frac{dy}{dx}
\end{aligned}$

}

%%%%%%%%%%%%%%%%%%%%%%%%%%%%%%%%%%%%%%%%%%%%


\frame
{
\textbf{Answer:}
\begin{equation} \nonumber
\frac{1}{2\sqrt{x}}
\end{equation}

}

%%%~~~~~~~~~~~~~~~~~~~~~~~~~~~~~~~~~~~~~~~~~

\frame
{
\textbf{Question :} 
$\begin{aligned}[t] 
y= \sqrt{3x-4}  \text{\ , \  \ find \ \  } \frac{dy}{dx}
\end{aligned}$
}

%%%%%%%%%%%%%%%%%%%%%%%%%%%%%%%%%%%%%%%%%%%%

\frame
{
\textbf{Answer:}
\begin{equation} \nonumber
\frac{3}{2\sqrt{3x-4}}
\end{equation}

}
%%%~~~~~~~~~~~~~~~~~~~~~~~~~~~~~~~~~~~~~~~~~
\frame
{
\textbf{Question :} 
$\begin{aligned}[t] 
y= \sqrt{\sin x}  \text{\ , \  \ find \ \  } \frac{dy}{dx}
\end{aligned}$
}

%%%%%%%%%%%%%%%%%%%%%%%%%%%%%%%%%%%%%%%%%%%%

\frame
{
\textbf{Answer:}
\begin{equation} \nonumber
\frac{\cos x}{2\sqrt{\sin x}}
\end{equation}

}

%%%%%%%%%%%%%%%%%%%%%%%%%%%%%%%%%%%%%%%%%%%%
%%%~~~~~~~~~~~~~~~~~~~~~~~~~~~~~~~~~~~~~~~~~



%%%~~~~~~~~~~~~~~~~~~~~~~~~~~~~~~~~~~~~~~~~~
\frame
{
\textbf{Question :} 
$\begin{aligned}[t] 
y= \sqrt{tan x}  \text{\ , \  \ find \ \  } \frac{dy}{dx}
\end{aligned}$
}

%%%%%%%%%%%%%%%%%%%%%%%%%%%%%%%%%%%%%%%%%%%%

\frame
{
\textbf{Answer:}
\begin{equation} \nonumber
\frac{sec^2x}{2\sqrt{tan x}}
\end{equation}

}

%%%%%%%%%%%%%%%%%%%%%%%%%%%%%%%%%%%%%%%%%%%%
%%%~~~~~~~~~~~~~~~~~~~~~~~~~~~~~~~~~~~~~~~~~



%%%~~~~~~~~~~~~~~~~~~~~~~~~~~~~~~~~~~~~~~~~~
\frame
{
\textbf{Question :} 
$\begin{aligned}[t] 
y= 3\sqrt{2}\sin x  \text{\ , \  \ find \ \  } \frac{dy}{dx}
\end{aligned}$
}

%%%%%%%%%%%%%%%%%%%%%%%%%%%%%%%%%%%%%%%%%%%%

\frame
{
$\begin{aligned}[t] 
 \frac{dy}{dx}= 3\sqrt{2}\cos x 
\end{aligned}$

}

%%%%%%%%%%%%%%%%%%%%%%%%%%%%%%%%%%%%%%%%%%%%
%%%~~~~~~~~~~~~~~~~~~~~~~~~~~~~~~~~~~~~~~~~~

%%%~~~~~~~~~~~~~~~~~~~~~~~~~~~~~~~~~~~~~~~~~
\frame
{
\textbf{Question :} 
$\begin{aligned}[t] 
y= \sqrt{x^2 + \sqrt{x^2 +1}}  \text{\ , \  \ find \ \  } \frac{dy}{dx}
\end{aligned}$
}

%%%%%%%%%%%%%%%%%%%%%%%%%%%%%%%%%%%%%%%%%%%%
\large
\frame
{
\textbf{Answer:}
\begin{align*} \nonumber
&= \frac{1}{2\sqrt{x^2 + \sqrt{x^2 +1}}} \frac{d }{dx}\left( {x^2 + \sqrt{x^2 +1}} \right) \\
&= \frac{1}{2\sqrt{x^2 + \sqrt{x^2 +1}}} \left( {2x + \frac{2x}{2\sqrt{x^2 +1}}} \right) \\
\end{align*}

}

%%%%%%%%%%%%%%%%%%%%%%%%%%%%%%%%%%%%%%%%%%%%
%%%~~~~~~~~~~~~~~~~~~~~~~~~~~~~~~~~~~~~~~~~~


%%%~~~~~~~~~~~~~~~~~~~~~~~~~~~~~~~~~~~~~~~~~
\frame
{
\textbf{Question :} 
$\begin{aligned}[t] 
y= \sqrt{tan \sqrt{x} }  \text{\ , \  \ find \ \  } \frac{dy}{dx}
\end{aligned}$
}

%%%%%%%%%%%%%%%%%%%%%%%%%%%%%%%%%%%%%%%%%%%%

\frame
{
\textbf{Answer:}
\begin{equation} \nonumber
\frac{1}{2\sqrt{tan \sqrt{x} }}
\end{equation}

}

%%%%%%%%%%%%%%%%%%%%%%%%%%%%%%%%%%%%%%%%%%%%
%%%~~~~~~~~~~~~~~~~~~~~~~~~~~~~~~~~~~~~~~~~~




\end{document}