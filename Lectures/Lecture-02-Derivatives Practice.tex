\documentclass{beamer} 
\usetheme{Berlin}
\usepackage{showexpl} 
\usepackage{tikz}
\usepackage{chemfig}
\usepackage[utf8]{inputenc}
\usepackage{cancel}
\usepackage{circuitikz}
\usepackage[english]{babel}
\usepackage{blindtext}
\usepackage{multicol}
\usepackage{ragged2e}



\title{Derivatives}
\subtitle{Implicit and Logarithmic}
\author{Milind Oka}



\lstloadlanguages{[LaTeX]Tex} 
\lstset{% 
     basicstyle=\ttfamily\small, 
     commentstyle=\itshape\ttfamily\small, 
     showspaces=false, 
     showstringspaces=false, 
     breaklines=true, 
     breakautoindent=true, 
     captionpos=t 
} 

\begin{document} 

\frame {
		\titlepage
	}

%%%%%%%%%%%%%% Start Slides %%%%%%%%%%%%% 

%%%%%%%%%%%%%%%%%%%%%%%%%%%%%%%%%%%%%%%%%%%%

\frame
{
\textbf{Question :} 
$\begin{aligned}[t] 
\text{If \ } x + \sqrt{xy} + y = 1 \text{\ , \  find \ \  } \frac{dy}{dx}
\end{aligned}$

\textbf{Answer:}
\begin{equation} \nonumber
\begin{alignedat}{4}
& \text{Given \ } x + \sqrt{xy} + y = 1 \\
& \text{Differentiating w.r.t. x \ }\\
& 1 + \frac{1}{2\sqrt{xy}} \left( x\frac{dy}{dx}+y \right)+\frac{dy}{dx}=0 \\
& \therefore 1 + \frac{x}{2\sqrt{xy}} \frac{dy}{dx}+ \frac{y}{2\sqrt{xy}}+\frac{dy}{dx}=0 \\
& \therefore \left(\frac{x}{2\sqrt{xy}}+1\right) \frac{dy}{dx}=-1-\frac{y}{2\sqrt{xy}}\\
\end{alignedat}
\,
\vrule
\, 
\begin{alignedat}{4}
& \therefore \left(\frac{x+2\sqrt{xy}}{2\sqrt{xy}}\right) \frac{dy}{dx}=-\frac{2\sqrt{xy}+y}{2\sqrt{xy}}\\
& \therefore  (x+2\sqrt{xy}\,)\ \frac{dy}{dx}= -2\sqrt{xy}-y \\
& \therefore \frac{dy}{dx} =-\,\frac{x+2\sqrt{xy}}{2\sqrt{xy}+y}  \\
& \\
& \\
\end{alignedat}
\end{equation}

}

%%%%%%%%%%%%%%%%%%%%%%%%%%%%%%%%%%%%%%%%%%%%

\frame
{ \textbf{Question :} 
$\begin{aligned}[t] 
\text{If \ } x^p y^q = (x+y)^{p+q} \text{\ , \ the show that \ \  } \frac{dy}{dx}=\frac{y}{x}
\end{aligned}$

%----------------------------------------
\textbf{Answer:}
\begin{equation} \nonumber
\begin{alignedat}{4}
& \text{Given \ } x^py^q = (x+y)^{p+q} \\
& \therefore \log (x^py^q) = \log (x+y)^{p+q} \\
& \therefore p\log x + q\log y) = (p+q)\log (x+y) \\
& \text{Differentiating w.r.t. x \ }\\
& \therefore \frac{p}{x}+\frac{q}{y}\frac{dy}{dx} = \frac{p+q}{x+y}\left( 1+\frac{dy}{dx}\right)\\
& \therefore \frac{p}{x}+\frac{q}{y}\frac{dy}{dx} = \frac{p+q}{x+y} + \frac{p+q}{x+y}\,\frac{dy}{dx}\\
& \therefore \left( \frac{q}{y}-\frac{p+q}{x+y} \right)  \frac{dy}{dx} = \frac{p+q}{x+y} - \frac{p}{x}\\
\end{alignedat}
\,
%\vrule
%\, 
\begin{alignedat}{4}
\end{alignedat}
\end{equation}


}

%%%%%%%%%%%%%%%%%%%%%%%%%%%%%%%%%%%%%%%%%%%%


%%%%%%%%%%%%%%%%%%%%%%%%%%%%%%%%%%%%%%%%%%%%
\frame
{ \textbf{Question :} 
$\begin{aligned}[t] 
\text{If \ } x^p y^q = (x+y)^{p+q} \text{\ , \ the show that \ \  } \frac{dy}{dx}=\frac{y}{x}
\end{aligned}$

%----------------------------------------
\textbf{Answer: Contd}

\begin{equation} \nonumber
\begin{alignedat}{4}
& \therefore \frac{q(x+y) - (p+q)y}{y(x+y)} \,  \frac{dy}{dx} = \frac{(p+q)x - p(x+y)}{(x+y)x}\\
& \therefore \frac{qx + qy - py - qy}{y} \,  \frac{dy}{dx} = \frac{px + qx - px - py}{x}\\
& \therefore \frac{\cancel{qx - py}}{y} \,  \frac{dy}{dx} =  \frac{\cancel{qx - py}}{x}\\
& \therefore \frac{dy}{dx}=\frac{y}{x}\\
& \\
& \\
& \\
& \\
& \\
& \\
\end{alignedat}
\end{equation}

}

%%%%%%%%%%%%%%%%%%%%%%%%%%%%%%%%%%%%%%%%%%%%%%%%%%%%%%%%%%%%%%%%




\end{document}