\documentclass[17pt]{extarticle}
\usepackage[a4paper, total={7.5in, 10.5in}]{geometry}
\usepackage[utf8]{inputenc}
\usepackage[english]{babel}
\usepackage{paralist} 
\usepackage{multicol}
\usepackage{amsmath}
\usepackage{amssymb}
\usepackage{enumitem}
\usepackage{textcomp}






\newcommand{\degree}{$^{\circ}\ $} %added space after degree

\setlength{\columnsep}{0.5cm}

\begin{document}
\centering 
{\large \bf XI Practicals\par}
\vspace{1cm}
\begin{multicols}{2}

%%%////////////////////////////////////
\section{Angle and its Measurement}
\noindent
\begin{enumerate}

\item Convert the following degree measures
in the radian measures : i) 70\degree ii) 120\degree iii) $\left(\frac{1}{4}\right)$\degree

\item The measures of the angles of the
triangle are in A. P. The smallest angle
is 40. Find the angles of the triangle in
degree and in radians.

\end{enumerate} 

%%%////////////////////////////////////
\section{Trigonometry - I}
\noindent
\begin{enumerate}
\item If $\tan \theta + \frac{1}{\tan \theta}=2$ then find the value of $\tan^2\theta + \frac{1}{\tan^2\theta}$ 

\item  Prove that $(\sec A - \tan A)^2=\frac{1-\sin A}{1+\sin A}$
\end{enumerate}

%%%////////////////////////////////////
\section{Trigonometry - II}
\noindent
\begin{enumerate}
\item Find the value of cos 15\degree

\item Prove that tan 20\degree tan 40\degree tan 60\degree tan 80\degree
= 3 

\end{enumerate} 

%%%////////////////////////////////////
\section{ Determinants and Matrices}
\noindent
\begin{enumerate}
\item Solve $
x+y+z = 6,  x-y+z = 2,  x+2y–z = 2$ using Cramer's Rule.

\item Solve $x +y - z = 1,  8x +3y - 6z = 1, -4x - y + 3z = 1$ using Cramer's Rule.
\end{enumerate} 

%%%////////////////////////////////////
\section{Straight Line}
\noindent
\begin{enumerate}
\item Find equations of lines which pass
through the origin and make an angle of 45\degree
with the line $3x - y = 6.$

\item Find the equation of line which passes
	 through the point of intersection of lines
	 $3 x + 2 y - 6 = 0 ,\; x + y + 1 = 0$ and the point A(2,1).

\end{enumerate} 

%%%////////////////////////////////////
\section{Circle}
\noindent
\begin{enumerate}

\item Find the equation of a circle whose centre is
(−3, 1) and which pass through the point (5, 2).

\item Find the equation of the tangent to the circle $x^2 + y^2 - 4x - 6y - 12 = 0$ at (-1, -1)

\end{enumerate} 

%%%////////////////////////////////////
\section{Conic Sections}
\noindent
\begin{enumerate}
\item

\end{enumerate} 

%%%////////////////////////////////////
\section{Measures of Dispersion}
\noindent
\begin{enumerate}
\item 

\end{enumerate} 

%%%////////////////////////////////////
\section{Probability}
\noindent
\begin{enumerate}
\item

\end{enumerate} 

%%%////////////////////////////////////
\section{Complex Numbers}
\noindent
\begin{enumerate}
\item 

\end{enumerate} 

%%%////////////////////////////////////
\section{Sequences and Series}
\noindent
\begin{enumerate}
\item

\end{enumerate} 

%%%////////////////////////////////////
\section{Permutations and Combination}
\noindent
\begin{enumerate}
\item 

\end{enumerate} 


%%%////////////////////////////////////
\section{Methods of Induction and Binomial Theorem}
\noindent
\begin{enumerate}
\item

\end{enumerate} 

%%%////////////////////////////////////
\section{Sets and Relations}
\noindent
\begin{enumerate}
\item 

\end{enumerate} 

%%%////////////////////////////////////
\section{Functions}
\noindent
\begin{enumerate}
\item

\end{enumerate} 

%%%////////////////////////////////////
\section{Limits}
\noindent
\begin{enumerate}
\item 

\end{enumerate} 

%%%////////////////////////////////////
\section{Continuity}
\noindent
\begin{enumerate}
\item

\end{enumerate} 

%%%////////////////////////////////////
\section{Differentiation}
\noindent
\begin{enumerate}
\item 

\end{enumerate} 


\end{multicols}
 
\end{document}