\documentclass[14pt]{article}
\usepackage[a4paper, total={7.5in, 10.5in}]{geometry}
\usepackage[utf8]{inputenc}
\usepackage[english]{babel}
\usepackage{paralist} 
\usepackage{multicol}
\usepackage{amsmath}
\usepackage{amssymb}
\usepackage{enumitem}
\usepackage{textcomp}


\newcommand{\degree}{$^{\circ}\ $} %added space after degree

\setlength{\columnsep}{0.5cm}

\begin{document}
\centering Mathematics Question Bank From 2019-20 onwards
\begin{multicols}{2}
%%/////////////////////////////////

\section{Angles and Their Measurement}
\noindent
\begin{enumerate}
  
\item Determine which of the following pairs
of angles are co-terminal i) 210\degree, 150\degree
ii) 360\degree, -30\degree iii) -180\degree, 540\degree iv) -405\degree, 675\degree v) 860\degree, 580\degree vi) 900\degree, -900\degree 
\item Draw the angles of the following measures
and determine their quadrants. i) –140\degree  ii) 250\degree iii) 420\degree iv) 750\degree v) 945\degree vi) 1120\degree vii) –80\degree viii) –330\degree ix) –500\degree x) –820\degree

\item Convert the following angles in to radian. i) 85\degree ii) 250\degree iii) -132\degree iv) 65\degree30' v) 75\degree30' vi) 40\degree48'

\item Convert the following angles in degree 
i) $ \frac{7\pi}{12} ^c $
ii) $ \frac{-5\pi}{3} ^c $
iii) $ 5^c $
iv) $ \frac{11\pi}{18} ^c $
v) $ \left( \frac{-1}{4}\right)^c $

\item The sum of two angles is $5\pi^c$ and their
difference is 60\degree. Find their measures in
degree.

\item Find the length of an arc of a circle
which subtends an angle of 108\degree at the
centre, if the radius of the circle is 15
cm.

\item Find the angle in degree subtended at
the centre of a circle by an arc whose
length is 15 cm, if the radius of the
circle is 25 cm.

\item The area of a circle is 25$\pi$ sq.cm. Find
the length of its arc subtending an angle
of 144\degree at the centre. Also find the area
of the corresponding sector.

\item The perimeter of a sector of the circle
of area 25$\pi$ sq.cm is 20 cm. Find the
area of sector.

\end{enumerate} 


%%%////////////////////////////////////
\section{Trigonometry-I}
\noindent
\begin{enumerate}[resume]

\item State the signs of
i) tan 380\degree
ii) cot 230\degree
iii) sec 468\degree

\item Evaluate each of the following :
i) sin30\degree + cos 45\degree + tan180\degree
ii) cosec45\degree + cot 45\degree + tan0\degree
iii) sin30\degree × cos 45\degree × tan360\degree

\item If $cos\;\theta = \frac{12}{13},\ \ 0<\theta<\frac{\pi}{2}$, find the value of i) $\frac{sin^2\theta-cos^2\theta}{2sin\;\theta\;cos\theta}\; \ , \frac{1}{tan^2\theta}$

\item Using tables evaluate the following : i) $4cot\;45$\degree $-sec^260$\degree $ + sin\;30$\degree 
ii) $cos^20+cos^2\frac{\pi}{6}+cos^2\frac{\pi}{3}+cos^2\frac{\pi}{2}$

\item If $\frac{sin\,A}{3}=\frac{sin\,B}{4}=\frac{1}{5} $ and A, B are angles in the second quadrant then prove that $4cos\,A+3cos\,B=-5$

\item If$tan\,\theta=\frac{1}{2}$, evaluate $\frac{2sin\,\theta+3cos\,\theta}{4cos\,\theta+3sin\,\theta}$

\item Eliminate $\theta$ from i) $x=3sec\,\theta,\;y=4tan\,\theta.$ ii) $x=6cosec\,\theta,\;y=8cot\,\theta.$

\item Find the acute angle $\theta$ such that $5tan^2\theta+3=9sec\,\theta.$

\item If $2cos^2\theta - 11cos\,\theta+5=0$ then find the possible values of $cos\,\theta$

\item Find the Cartesian co-ordinates of points
whose polar coordinates are : i) (3,90$^{\circ}$) ii) (1,180$^{\circ}$)

\item Prove the following identities: i) $(cos^A-1)(cot^A+1)=-1$ ii) $\frac{sin\,\theta}{1+cos\,\theta}+\frac{1+cos\,\theta}{sin\,\theta}=2cos\,\theta$ iii) $\frac{tan\,\theta}{sec\,\theta-1}=\frac{sec\,\theta+1}{tan\,\theta}$ iv) $(sec\,A+cos\,A)(sec\,A-cos\,A)=tan^2A+sin^2A$

\end{enumerate} 




%%%////////////////////////////////////
\section{Trigonometry-II}
\noindent
\begin{enumerate}[resume]

\item Find the values of i) sin 15\degree ii) cos 75\degree iii) tan 105\degree iv) cot 225\degree

\item Prove the following : i) $tan \left(\frac{\pi}{4}+\theta \right)=\frac{1-tan\,\theta}{1+tan\,\theta}$ ii) $\sqrt{2}cos \left(\frac{\pi}{4}-A \right) = cos A + sin A$ iii) tan 50\degree = tan 40\degree+2tan 10\degree

\item If $tan\,A=\frac{5}{6},\,tan\,B=\frac{1}{11}$, prove that $A+B=\frac{\pi}{4} $  

\item Find the value of i) sin 690\degree ii) cos 315\degree iii) tan 225\degree


\item Prove the following : i) $\frac{cos\,(\pi+x)cos(-x)}{sin(\pi-x)cos\left(\frac{\pi}{2}+x \right)}=cot^2x$ ii) sec 840\degree.cot (- 945$^{\circ}$) + sin 600\degree tan (-690$^{\circ}$)$=\frac{3}{2}$

\item Prove the following : i) $\frac{1-cos\,2\theta}{1+cos\,2\theta}=tan^2\theta$ ii) $tan\,x + cot\,x = 2\,cosec\,2x $

\end{enumerate} 



%%%////////////////////////////////////
\section{Determinants and Matrices}
\noindent
\begin{enumerate}[resume]

\item Find the value of 
i) $\begin{vmatrix} \,a & b \\c & d \,\end{vmatrix}$ 
ii) $\begin{vmatrix} 3 & -4 & 5 \\ \, 1 & 1 & -2\,\\ 2 & 3 & 1\end{vmatrix}$ 

\item Find x.
i) $\begin{vmatrix} x^2-x+1 & x+1 \\x+1 & x+1 \,\end{vmatrix}$=0 
ii) $\begin{vmatrix} x & -1 & 2 \\ 2x & 1 & 3\,\\ 3 & -4 & 5\end{vmatrix}=29$ 

\item Find minor and cofactors of
 $\begin{vmatrix} 2 & -1 & 3 \\ 1 & 2 & -1\,\\ 5 & 7 & 2\end{vmatrix}$ 

\item Using properties show that
 $\begin{vmatrix} a+b & a & b \\ a & a+c & c\,\\ b & c & b+c\end{vmatrix}$= 4abc

\item Solve : 
 $\begin{vmatrix} x+2 & x+6 & x-1 \\ x+6 & x-1 & x+2\,\\ x-1 & x+2 & x+6\end{vmatrix}= 0$

\item Solve by Cramer's Rule :  $x+y+z = 6, x–y+z = 2, x+2y–z = 2$

\item Find k if the following matrices are singular. 
i) $\begin{bmatrix} 7 & 3 \\-2 & k \,\end{bmatrix}$
ii) $\begin{bmatrix} x+2 & x+6 & x-1 \\ x+6 & x-1 & x+2\,\\ x-1 & x+2 & x+6\end{bmatrix}$
 
\item Find a, b, c if the matrix $\begin{bmatrix} 1 & \frac{3}{5} & a \\ b & -5 & 7\,\\ -4 & c & 0\end{bmatrix}$ is symmetric.

\item Solve for $X$ and $Y$ : $3X-Y=\begin{bmatrix} 1 & -1 \\-1 & 1\end{bmatrix}$ and $X-3Y=\begin{bmatrix} 0 & -1 \\0 & -1\end{bmatrix}$ 

\item If $A =\begin{bmatrix} 4 & 8 \\-2 & -4\end{bmatrix}$, prove that $A^2=0.$

\item If $A=\begin{bmatrix} 1 & 2 & 2 \\ 2 & 1 & 2\,\\ 2 & 2 & 1\end{bmatrix}$, show that $A^2-4A$ is a scalar matrix.

\item If $A =\begin{bmatrix} 3 & 4 \\-4 & 3\end{bmatrix}$, prove that $A^2=0.$ and $B =\begin{bmatrix} 2 & 1 \\-1 & 2\end{bmatrix}$, show that $(A+B)(A-B)=A^2-B^2.$

\end{enumerate} 






%%%////////////////////////////////////
\section{Straight Line}
\noindent
\begin{enumerate}[resume]

\item If A(2, 0) and B(0, 3) are two points, find
the equation of the locus of point P such
that AP = 2BP.

\item If A(4, 1) and B(5, 4), find the equation
of the locus of point P if PA\textsuperscript{2} = 3PB\textsuperscript{2}.

\item Obtain the new equations of the following
loci if the origin is shifted to the point
O'(2, 2), the direction of axes remaining
the same : (a) $3x -y+2=0$ (b) $x^2+y^2-3x=7$

\item A line makes intercepts 3 and 3 on the
co-ordiante axes. Find the inclination of
the line.

\item Without using Pythagoras theorem show
that points A(4,4), B(3, 5) and C(1, 1)
are the vertices of a right angled triangle.

\item Find the value of k for which points
P(k,1),Q(2,1) and R(4,5) are collinear.

\item Find the equation of the line a) passing through the points A ( 2 , 0 ) and
B(3,4). b) passing through the points P(2,1) and
Q(2,-1)

\item Find the equation of the line a) containing the
origin and having inclination 60\degree. b) passing through the origin and parallel to AB, where A is (2,4) and B is (1,7). c) having slope $\frac{1}{2}$
and containing the point (3,2).

\item The vertices of a triangle are A(3,4), B(2,0)
and C(1,6). Find the equations of the lines containing (a) side BC (b) the median AD (c) the mid points of sides AB and BC.

\item Show that lines $x-2y-7=0$ and $2x-4y+15=0$ are parallel to each other.

\item Show that lines $x - 2 y - 7 =0$ and 
$2x + y + 1 = 0$ are perpendicular to each
other. Find their point of intersection.

\item If the line $3 x + 4 y = p$ makes a triangle
of area 24 square unit with the co-ordinate
axes then find the value of $p$.

\item Find the distance of the point A (-2,3) from the line $12 x - 5 y - 13 = 0$

\item Find the distance between parallel lines
$4x - 3 y + 5 = 0$ and $4x - 3y + 7 = 0$,


\end{enumerate} 






%%%////////////////////////////////////
\section{Permutations and Combinations}
\noindent
\begin{enumerate}[resume]
  
\item How many two letter words can be formed
using letters from the word SPACE, when
repetition of letters (i) is allowed, (ii) is not
allowed?
\item How many three-digit numbers can be
formed from the digits 0, 1, 3, 5, 6 if
repetitions of digits (i) are allowed, (ii) are
not allowed?
\item How many numbers between 100 and 1000
have 4 in the units place?
		 
\item Write in terms of factorials
(i)	$5 \times 6 \times 7 \times 8 \times 9 \times 10	$
(ii)	$3 \times 6 \times 9 \times 12 \times 15$
(iii)	$6 \times 7 \times 8 \times 9$
(iv)  $5 \times 10 \times 15 \times 20$

\item 
Evaluate : $\frac{n!}{r!(n-r)!}$ for  (i)	 n = 8, r = 6	 (ii)	 n = 12, r = 12,
	 (iii)	 n = 15, r = 10	 (iv)	 n = 15, r = 8
	 
\item Find n if
(i) $\frac{n}{8!}= \frac{3}{6!}+\frac{1!}{4!}$
(ii) $\frac{n}{6!}= \frac{4}{8!}+\frac{3}{6!}$
(iii) $\frac{1!}{n!}= \frac{1!}{4!}+\frac{4}{5!}$
(iv) $(n+1)!=42 \times (n-1)!$
(v) $(n+3)!=110 \times (n+1)!$

\item Find n if
(i) $\frac{(17-n)!}{(14-n)!}= 5!$
(ii) $\frac{(15-n)!}{(13-n)!}= 12$
(iii) $\frac{n!}{3!(n-3)!}:\frac{n!}{5!(n-5)!}=5:3$
(iv) $\frac{n!}{3!(n-3)!}:\frac{n!}{5!(n-7)!}=1:6$
(v) $\frac{(2n)!}{7!(2n-7)!}:\frac{n!}{4!(n-4)!}=21:1$ 

\item Find n if $^nP_6 : ^nP_3=120:1$

\item Find m and n, if $^{m+n}P_2 = 56 and ^{m+n}nP_2=12$

\item Find r if $^12P_{r-2} : ^11P_{r-1}=3:14$

\item Find the number of permutations of the
letters of the word UBUNTU.

\item How many 4 letter words can be formed
using letters in the word MADHURI if
(a) letters can be repeated (b) letters cannot
be repeated.



\item Determine the number of arrangements of
letters of the word ALGORITHM if
(a) vowels are always together.
(b) no two vowels are together.
(c) consonants are at even positions.

\item Find the number of arrangements of the letters
in the word SOLAPUR so that consonents
and vowels are placed alternately.

\item Find the value of  (a) $ ^{16}C_4 $
(b)  $ ^{80}C_2$
(c)  $ ^{15}C_4 + ^{15}C_5 $
(d)  $ ^{20}C_{16} - ^{19}C_{16} $

\item Find n if  (a) $ ^6P_2 = n ^6C_2 $
(b)  $ ^{2n}C_3 : ^nC_2 = 52:3$
(c)  $ ^nC_{n-3}=84$

\item Find the number of ways of selecting a team
of 3 boys and 2 girls from 6 boys and 4 girls.

\item Find r if  $ ^{11}C_4 + ^{11}C_5 + ^{12}C_6 +  ^{13}C_7=^{14}C_r$
 
 \item A group consists of 9 men and 6 women.
A team of 6 is to be selected. How many
of possible selections will have at least
3 women?
\end{enumerate} 


\section{Set Theory}
\noindent
\begin{enumerate}[resume]

\item  If A = \{ 1, 2, 3, 4\}, B = \{3, 4, 5, 6\}
		  C = \{4, 5, 6, 7, 8\} and universal set  X = \{1, 2, 3, 4, 5, 6, 7, 8, 9, 10\}, then verify
the following:
i)		$ A \cup B\cap C) = (A\cup B) \cap (A \cup C)$

ii)	$A\cap (B\cup C)=(A\cap B)\cup (A \cup C)$
iii) $(A\cup B)' = (A'\cap B)'         $
iv)	$ (A\cap B)' = A'\cup B'          $
v)	$	 A = (A\cap B)\cup (A\cap B') $
vi)	$	 B = (A\cap B)\cup (A'\cap B) $
vii) $(A\cup B)=(A-B)\cup (A\cap B)\cup (B-A)$
viii)	$A \cap (B\Delta C) = (A\cap B) \Delta (A\cap C)$
ix) $n (A\cup B) = n(A) + n(B) - n(A\cap B)$ 
x) $n (B) = n(A'\cap B) + n(A\cap B)$

\item If A and B are subsets of the universal set X and n(X) = 50, n(A) = 35, n(B) = 20,$ n(A'\cap B') = 5,$ find i) $n (A\cup B)$ ii) $n(A\cap B)$
iv) $ n(A\cap B')$
iii) $ n(A'\cap B) $

\item In a class of 200 students who appeared
certain examinations, 35 students failed
in CET, 40 in NEET and 40 in JEE,
20 failed in CET and NEET, 17 in NEET
and JEE, 15 in CET and JEE and 5 failed
in all three examinations. Find how many
students,
i)	did not fail in any examination.
ii) failed in NEET or JEE entrance.
		 
		 
\item In a hostel, 25 students take tea, 20 students
take coffee, 15 students take milk, 10
student take bot tea and coffee, 8 students
take both milk and coffee. None of them
take tea and milk both and everyone takes
atleast one beverage, find the total number
of students in the hostel.

\item If $P = \{1, 2, 3 \} $ and $Q = \{14 \}$,find sets $P \times Q $ and $Q \times P$

\item Let A = \{1, 2, 3, 4\}, B = \{4, 5, 6\}, C = \{ 5, 6 \}.
 Verify, i) $A \times (B \cap C) = (A \times B) \cap (A \times C)$				
ii) $ A \times (B \cup C) = (A \times B) \cup (A \times C)$

\item Let A = \{6, 8\} and B = \{1, 3, 5\} Show that $ R_1 = \{ (a, b) / a \in A, b\in B, a - b$
is an even number \} is a null relation.
$R_2 = \{(a, b)/a \in A, b \in B$, a+b is odd number \}
is an universal relation.
\end{enumerate} 



\end{multicols}
 
\end{document}