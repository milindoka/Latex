\documentclass[14pt]{article}
\usepackage[a4paper, total={7in, 8in}]{geometry}
\usepackage[utf8]{inputenc}
\usepackage[english]{babel}
\usepackage{paralist} 
\usepackage{multicol}
\setlength{\columnsep}{1cm}

\begin{document}
\begin{multicols}{2}
[
\section{Set Theory}
Note : For Terminal Exam 4 Questions of 3 marks each will be asked
]
\noindent
\begin{enumerate}
  \item In a class of 200 students who appeared
certain examinations, 35 students failed
in CET, 40 in NEET and 40 in JEE,
20 failed in CET and NEET, 17 in NEET
and JEE, 15 in CET and JEE and 5 failed
in all three examinations. Find how many
students,
		 i)	 did not fail in any examination.
		 ii)	failed in NEET or JEE entrance.
		 
\item  If A = \{ 1, 2, 3, 4\}, B = \{3, 4, 5, 6\}
		  C = \{4, 5, 6, 7, 8\} and universal set
		 X = \{1, 2, 3, 4, 5, 6, 7, 8, 9, 10\}, then verify
the following:
i)		 A%\cup B∩C) = (A∪B) ∩ (A∪C)
\iffalse
ii)		 A∩(B∪C) = (A∩B) ∪ (A∪C)
iii)	(A∪B)' = (A'∩B)'
iv)	(A∩B)' = A'∪B'
v)		 A = (A∩B)∪ (A∩B')
vi)		 B = (A∩B)∪ (A'∩B)
vii)		(A∪B) = (A−B) ∪ (A∩B) ∪ (B−A)
Since n(A∩B)≤n(A), n(A∩B)≤n(B), then
viii)		A ∩ (B∆C) = (A∩B) ∆ (A∩C)
ix)	n (A∪B) = n(A) + n(B) - n(A∩B)
x)	 n (B) = (A'∩B) + n(A∩B) 
\fi
\end{enumerate}
\end{multicols}
 
\end{document}