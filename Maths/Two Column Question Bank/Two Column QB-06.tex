\documentclass[14pt]{article}
\usepackage[a4paper, total={7.5in, 8in}]{geometry}
\usepackage[utf8]{inputenc}
\usepackage[english]{babel}
\usepackage{paralist} 
\usepackage{multicol}
\usepackage{amsmath}
\usepackage{amssymb}
\setlength{\columnsep}{0.5cm}

\begin{document}
\begin{multicols}{2}

\section{Permutations and Combinations}
\noindent
\begin{enumerate}
  
\item How many two letter words can be formed
using letters from the word SPACE, when
repetition of letters (i) is allowed, (ii) is not
allowed?
\item How many three-digit numbers can be
formed from the digits 0, 1, 3, 5, 6 if
repetitions of digits (i) are allowed, (ii) are
not allowed?
\item How many numbers between 100 and 1000
have 4 in the units place?
		 
\item Write in terms of factorials
(i)	$5 \times 6 \times 7 \times 8 \times 9 \times 10	$
(ii)	$3 \times 6 \times 9 \times 12 \times 15$
(iii)	$6 \times 7 \times 8 \times 9$
(iv)  $5 \times 10 \times 15 \times 20$

\item 
Evaluate : $\frac{n!}{r!(n-r)!}$ for  (i)	 n = 8, r = 6	 (ii)	 n = 12, r = 12,
	 (iii)	 n = 15, r = 10	 (iv)	 n = 15, r = 8
	 
\item Find n if
(i) $\frac{n}{8!}= \frac{3}{6!}+\frac{1!}{4!}$
(ii) $\frac{n}{6!}= \frac{4}{8!}+\frac{3}{6!}$
(iii) $\frac{1!}{n!}= \frac{1!}{4!}+\frac{4}{5!}$
(iv) $(n+1)!=42 \times (n-1)!$
(v) $(n+3)!=110 \times (n+1)!$

\item Find n if
(i) $\frac{(17-n)!}{(14-n)!}= 5!$
(ii) $\frac{(15-n)!}{(13-n)!}= 12$
(iii) $\frac{n!}{3!(n-3)!}:\frac{n!}{5!(n-5)!}=5:3$
(iv) $\frac{n!}{3!(n-3)!}:\frac{n!}{5!(n-7)!}=1:6$
(v) $\frac{(2n)!}{7!(2n-7)!}:\frac{n!}{4!(n-4)!}=21:1$ 

\item Find n if $^nP_6 : ^nP_3=120:1$

\item Find m and n, if $^{m+n}P_2 = 56 and ^{m+n}nP_2=12$

\item Find r if $^12P_{r-2} : ^11P_{r-1}=3:14$

\item Find the number of permutations of the
letters of the word UBUNTU.

\item How many 4 letter words can be formed
using letters in the word MADHURI if
(a) letters can be repeated (b) letters cannot
be repeated.



\item Determine the number of arrangements of
letters of the word ALGORITHM if
(a) vowels are always together.
(b) no two vowels are together.
(c) consonants are at even positions.

\item Find the number of arrangements of the letters
in the word SOLAPUR so that consonents
and vowels are placed alternately.

\end{enumerate} 


\section{Set Theory}
\noindent
\begin{enumerate}
  
\item  If A = \{ 1, 2, 3, 4\}, B = \{3, 4, 5, 6\}
		  C = \{4, 5, 6, 7, 8\} and universal set  X = \{1, 2, 3, 4, 5, 6, 7, 8, 9, 10\}, then verify
the following:
i)		$ A \cup B\cap C) = (A\cup B) \cap (A \cup C)$

ii)	$A\cap (B\cup C)=(A\cap B)\cup (A \cup C)$
iii) $(A\cup B)' = (A'\cap B)'         $
iv)	$ (A\cap B)' = A'\cup B'          $
v)	$	 A = (A\cap B)\cup (A\cap B') $
vi)	$	 B = (A\cap B)\cup (A'\cap B) $
vii) $(A\cup B)=(A-B)\cup (A\cap B)\cup (B-A)$
Since $n(A \cap B)  \leq n(A),n(A\cap B)\leq n(B)$, then
viii)	$A \cap (B\delta C) = (A\cap B) \delta (A\cap C)$
ix) $n (A\cup B) = n(A) + n(B) - n(A\cap B)$
x) $n (B) = (A'\cap B) + n(A\cap B)$

\item If A and B are subsets of the universal set X and n(X) = 50, n(A) = 35, n(B) = 20,$ n(A'\cap B') = 5,$ find i) $n (A\cup B)$ ii) $n(A\cap B)$
iv) $ n(A\cap B')$
iii) $ n(A'\cap B) $

\item In a class of 200 students who appeared
certain examinations, 35 students failed
in CET, 40 in NEET and 40 in JEE,
20 failed in CET and NEET, 17 in NEET
and JEE, 15 in CET and JEE and 5 failed
in all three examinations. Find how many
students,
i)	did not fail in any examination.
ii) failed in NEET or JEE entrance.
		 
		 
\item In a hostel, 25 students take tea, 20 students
take coffee, 15 students take milk, 10
student take bot tea and coffee, 8 students
take both milk and coffee. None of them
take tea and milk both and everyone takes
atleast one beverage, find the total number
of students in the hostel.

\item If $P = \{1, 2, 3 \} $ and $Q = \{14 \}$,find sets $P \times Q $ and $Q \times P$

\item Let A = \{1, 2, 3, 4\}, B = \{4, 5, 6\}, C = \{ 5, 6 \}.
 Verify, i) $A \times (B \cap C) = (A \times B) \cap (A \times C)$				
ii) $ A \times (B \cup C) = (A \times B) \cup (A \times C)$

\item Let A = \{6, 8\} and B = \{1, 3, 5\} Show that $ R_1 = \{ (a, b) / a \in A, b\in B, a - b$
is an even number \} is a null relation.
$R_2 = \{(a, b)/a \in A, b \in B$, a+b is odd number \}
is an universal relation.
\end{enumerate} 





\end{multicols}
 
\end{document}