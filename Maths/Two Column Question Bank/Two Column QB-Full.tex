\documentclass[14pt]{article}
\usepackage[a4paper, total={7.5in, 10.5in}]{geometry}
\usepackage[utf8]{inputenc}
\usepackage[english]{babel}
\usepackage{paralist} 
\usepackage{multicol}
\usepackage{amsmath}
\usepackage{amssymb}
\usepackage{enumitem}
\usepackage{textcomp}


\newcommand{\degree}{$^{\circ}\ $} %added space after degree

\setlength{\columnsep}{0.5cm}

\begin{document}
\centering 
{\huge \bf Mathematics QB From 2019-20 onwards\par}
\vspace{1cm}
\begin{multicols}{2}
%%/////////////////////////////////
\section{Angles and Their Measurement}
\noindent
\begin{enumerate}
  
\item Determine which of the following pairs
of angles are co-terminal i) 210\degree, 150\degree
ii) 360\degree, -30\degree iii) -180\degree, 540\degree iv) -405\degree, 675\degree v) 860\degree, 580\degree vi) 900\degree, -900\degree 
\item Draw the angles of the following measures
and determine their quadrants. i) –140\degree  ii) 250\degree iii) 420\degree iv) 750\degree v) 945\degree vi) 1120\degree vii) –80\degree viii) –330\degree ix) –500\degree x) –820\degree

\item Convert the following angles in to radian. i) 85\degree ii) 250\degree iii) -132\degree iv) 65\degree30' v) 75\degree30' vi) 40\degree48'

\item Convert the following angles in degree 
i) $ \frac{7\pi}{12} ^c $
ii) $ \frac{-5\pi}{3} ^c $
iii) $ 5^c $
iv) $ \frac{11\pi}{18} ^c $
v) $ \left( \frac{-1}{4}\right)^c $

\item The sum of two angles is $5\pi^c$ and their
difference is 60\degree. Find their measures in
degree.

\item Find the length of an arc of a circle
which subtends an angle of 108\degree at the
centre, if the radius of the circle is 15
cm.

\item Find the angle in degree subtended at
the centre of a circle by an arc whose
length is 15 cm, if the radius of the
circle is 25 cm.

\item The area of a circle is 25$\pi$ sq.cm. Find
the length of its arc subtending an angle
of 144\degree at the centre. Also find the area
of the corresponding sector.

\item The perimeter of a sector of the circle
of area 25$\pi$ sq.cm is 20 cm. Find the
area of sector.

\end{enumerate} 


%%%////////////////////////////////////
\section{Trigonometry-I}
\noindent
\begin{enumerate}[resume]

\item State the signs of
i) tan 380\degree
ii) cot 230\degree
iii) sec 468\degree

\item Evaluate each of the following :
i) sin30\degree + cos 45\degree + tan180\degree
ii) cosec45\degree + cot 45\degree + tan0\degree
iii) sin30\degree × cos 45\degree × tan360\degree

\item If $cos\;\theta = \frac{12}{13},\ \ 0<\theta<\frac{\pi}{2}$, find the value of i) $\frac{sin^2\theta-cos^2\theta}{2sin\;\theta\;cos\theta}\; \ , \frac{1}{tan^2\theta}$

\item Using tables evaluate the following : i) $4cot\;45$\degree $-sec^260$\degree $ + sin\;30$\degree 
ii) $cos^20+cos^2\frac{\pi}{6}+cos^2\frac{\pi}{3}+cos^2\frac{\pi}{2}$

\item If $\frac{sin\,A}{3}=\frac{sin\,B}{4}=\frac{1}{5} $ and A, B are angles in the second quadrant then prove that $4cos\,A+3cos\,B=-5$

\item If$tan\,\theta=\frac{1}{2}$, evaluate $\frac{2sin\,\theta+3cos\,\theta}{4cos\,\theta+3sin\,\theta}$

\item Eliminate $\theta$ from i) $x=3sec\,\theta,\;y=4tan\,\theta.$ ii) $x=6cosec\,\theta,\;y=8cot\,\theta.$

\item Find the acute angle $\theta$ such that $5tan^2\theta+3=9sec\,\theta.$

\item If $2cos^2\theta - 11cos\,\theta+5=0$ then find the possible values of $cos\,\theta$

\item Find the Cartesian co-ordinates of points
whose polar coordinates are : i) (3,90$^{\circ}$) ii) (1,180$^{\circ}$)

\item Prove the following identities: i) $(cos^A-1)(cot^A+1)=-1$ ii) $\frac{sin\,\theta}{1+cos\,\theta}+\frac{1+cos\,\theta}{sin\,\theta}=2cos\,\theta$ iii) $\frac{tan\,\theta}{sec\,\theta-1}=\frac{sec\,\theta+1}{tan\,\theta}$ iv) $(sec\,A+cos\,A)(sec\,A-cos\,A)=tan^2A+sin^2A$

\end{enumerate} 




%%%////////////////////////////////////
\section{Trigonometry-II}
\noindent
\begin{enumerate}[resume]

\item Find the values of i) sin 15\degree ii) cos 75\degree iii) tan 105\degree iv) cot 225\degree

\item Prove the following : i) $tan \left(\frac{\pi}{4}+\theta \right)=\frac{1-tan\,\theta}{1+tan\,\theta}$ ii) $\sqrt{2}cos \left(\frac{\pi}{4}-A \right) = cos A + sin A$ iii) tan 50\degree = tan 40\degree+2tan 10\degree

\item If $tan\,A=\frac{5}{6},\,tan\,B=\frac{1}{11}$, prove that $A+B=\frac{\pi}{4} $  

\item Find the value of i) sin 690\degree ii) cos 315\degree iii) tan 225\degree


\item Prove the following : i) $\frac{cos\,(\pi+x)cos(-x)}{sin(\pi-x)cos\left(\frac{\pi}{2}+x \right)}=cot^2x$ ii) sec 840\degree.cot (- 945$^{\circ}$) + sin 600\degree tan (-690$^{\circ}$)$=\frac{3}{2}$

\item Prove the following : i) $\frac{1-cos\,2\theta}{1+cos\,2\theta}=tan^2\theta$ ii) $tan\,x + cot\,x = 2\,cosec\,2x $

\end{enumerate} 



%%%////////////////////////////////////
\section{Determinants and Matrices}
\noindent
\begin{enumerate}[resume]

\item Find the value of 
i) $\begin{vmatrix} \,a & b \\c & d \,\end{vmatrix}$ 
ii) $\begin{vmatrix} 3 & -4 & 5 \\ \, 1 & 1 & -2\,\\ 2 & 3 & 1\end{vmatrix}$ 

\item Find x.
i) $\begin{vmatrix} x^2-x+1 & x+1 \\x+1 & x+1 \,\end{vmatrix}$=0 
ii) $\begin{vmatrix} x & -1 & 2 \\ 2x & 1 & 3\,\\ 3 & -4 & 5\end{vmatrix}=29$ 

\item Find minor and cofactors of
 $\begin{vmatrix} 2 & -1 & 3 \\ 1 & 2 & -1\,\\ 5 & 7 & 2\end{vmatrix}$ 

\item Using properties show that
 $\begin{vmatrix} a+b & a & b \\ a & a+c & c\,\\ b & c & b+c\end{vmatrix}$= 4abc

\item Solve : 
 $\begin{vmatrix} x+2 & x+6 & x-1 \\ x+6 & x-1 & x+2\,\\ x-1 & x+2 & x+6\end{vmatrix}= 0$

\item Solve by Cramer's Rule :  $x+y+z = 6, x–y+z = 2, x+2y–z = 2$

\item Find k if the following matrices are singular. 
i) $\begin{bmatrix} 7 & 3 \\-2 & k \,\end{bmatrix}$
ii) $\begin{bmatrix} x+2 & x+6 & x-1 \\ x+6 & x-1 & x+2\,\\ x-1 & x+2 & x+6\end{bmatrix}$
 
\item Find a, b, c if the matrix $\begin{bmatrix} 1 & \frac{3}{5} & a \\ b & -5 & 7\,\\ -4 & c & 0\end{bmatrix}$ is symmetric.

\item Solve for $X$ and $Y$ : $3X-Y=\begin{bmatrix} 1 & -1 \\-1 & 1\end{bmatrix}$ and $X-3Y=\begin{bmatrix} 0 & -1 \\0 & -1\end{bmatrix}$ 

\item If $A =\begin{bmatrix} 4 & 8 \\-2 & -4\end{bmatrix}$, prove that $A^2=0.$

\item If $A=\begin{bmatrix} 1 & 2 & 2 \\ 2 & 1 & 2\,\\ 2 & 2 & 1\end{bmatrix}$, show that $A^2-4A$ is a scalar matrix.

\item If $A =\begin{bmatrix} 3 & 4 \\-4 & 3\end{bmatrix}$, prove that $A^2=0.$ and $B =\begin{bmatrix} 2 & 1 \\-1 & 2\end{bmatrix}$, show that $(A+B)(A-B)=A^2-B^2.$

\end{enumerate} 


%%%////////////////////////////////////
\section{Straight Line}
\noindent
\begin{enumerate}[resume]

\item If A(2, 0) and B(0, 3) are two points, find
the equation of the locus of point P such
that AP = 2BP.

\item If A(4, 1) and B(5, 4), find the equation
of the locus of point P if PA\textsuperscript{2} = 3PB\textsuperscript{2}.

\item Obtain the new equations of the following
loci if the origin is shifted to the point
O'(2, 2), the direction of axes remaining
the same : (a) $3x -y+2=0$ (b) $x^2+y^2-3x=7$

\item A line makes intercepts 3 and 3 on the
co-ordiante axes. Find the inclination of
the line.

\item Without using Pythagoras theorem show
that points A(4,4), B(3, 5) and C(1, 1)
are the vertices of a right angled triangle.

\item Find the value of k for which points
P(k,1),Q(2,1) and R(4,5) are collinear.

\item Find the equation of the line a) passing through the points A ( 2 , 0 ) and
B(3,4). b) passing through the points P(2,1) and
Q(2,-1)

\item Find the equation of the line a) containing the
origin and having inclination 60\degree. b) passing through the origin and parallel to AB, where A is (2,4) and B is (1,7). c) having slope $\frac{1}{2}$
and containing the point (3,2).

\item The vertices of a triangle are A(3,4), B(2,0)
and C(1,6). Find the equations of the lines containing (a) side BC (b) the median AD (c) the mid points of sides AB and BC.

\item Show that lines $x-2y-7=0$ and $2x-4y+15=0$ are parallel to each other.

\item Show that lines $x - 2 y - 7 =0$ and 
$2x + y + 1 = 0$ are perpendicular to each
other. Find their point of intersection.

\item If the line $3 x + 4 y = p$ makes a triangle
of area 24 square unit with the co-ordinate
axes then find the value of $p$.

\item Find the distance of the point A (-2,3) from the line $12 x - 5 y - 13 = 0$

\item Find the distance between parallel lines
$4x - 3 y + 5 = 0$ and $4x - 3y + 7 = 0$,


\end{enumerate} 



%%%////////////////////////////////////
\section{Complex Numbers}
\noindent
\begin{enumerate}[resume]


\item Find a and b if i) $a + 2b + 2ai = 4 + 6i$ 
ii) $(a-b) + (a+b)i = a + 5i$
iii) $(a+b) (2 + i) = b + 1 + (10 + 2a)i$
iv) $\frac{1}{a+ib}=3-2i$

\item Express the following in the form of a+ib
i) $(1+2i)(-2+i)$ ii) $\frac{2+i}{(3-i)(1+2i)} $
iii) $\left( \frac{1+i}{1-i} \right)^2$ vii) $(1+i)^{-3}$

\item Show that $(-1+3i)^3$ is a real number.

\item Evaluate : i) $i^{35}$ ii) $i^{-888}$ iii) $i^{93}$ iv) $i^{116}$

\item Find the value of x and y if  $(x+2y) + (2x-3y) i + 4i = 5$

\item Find the square root of the following
complex numbers. i) $-8-6i$ ii) $7+24i$
iii) $1+4\sqrt{3}i$

\item Find the value of i) $x^3 - x^2 +x+46$, if $x = 2+3i$
 ii) $x^3 + x^2 -x+22$, if $x=\frac{5}{1-2i}$ iii) $x^ 3 - x^2 + 2x + 10$ if $1+\sqrt{3}i$ (Ans  , ,6)

\end{enumerate} 





%%%////////////////////////////////////
\section{Sequences and Series}
\noindent
\begin{enumerate}[resume]


\item Check  whether the following sequences are
G.P. If so, write $t_n$ : i) 2,6,18,54,... ii) 1,-5,25, -125 ... 

\item For a GP i) If $r=\frac{1}{3},a=9$, find $t_7$ ii) If $a=\frac{7}{243},r=3$ find $t_6$ iii ) If $r = - 3$ and $t_6$ = 1701, find a.

\item Find three numbers in G.P. such that their
sum is 21 and sum of their squares is 189.

\item For a GP i) If $a=2,r=-\frac{2}{3}$, find $S_6$ ii) If $S_5=1023,r=4$ find a  iii ) If $r = - 3$ and $t_6$ = 1701, find a.

\item For a G.P. sum of first 3 terms is 125
and sum of next 3 terms is 27,
find the value of  r .

\item Find the sum to n terms i) 3 + 33 + 333 + 3333 + ...  ii) 8 +88 + 888 + 8888 + ... iii) 0.4 + 0.44 + 0.444 + ... iv) 0.7 + 0.77+ 0.777 + ...

\item Find the n\textsuperscript{th} term and hence find the 8\textsuperscript{th} term
of the following HPs i) $\frac{1}{3},\frac{1}{5},\frac{1}{7},\frac{1}{9},$ ii) $\frac{1}{3},\frac{1}{6},\frac{1}{12},\frac{1}{24},$  

\item Find the sum i) $\sum\limits_{r=1}^n (r+1)(2r-1)$
$\sum\limits_{r=1}^n (3r^2-2r+1)$ iii) $\sum\limits_{r=1}^n \frac{1+2+3+..+r}{r}$\quad iv) $\sum\limits_{r=1}^n \frac{1^3+2^3+3^3+..+r^3}{r(r+1)}$
\end{enumerate} 


%%%////////////////////////////////////
\section{Permutations and Combinations}
\noindent
\begin{enumerate}[resume]
  
\item How many two letter words can be formed
using letters from the word SPACE, when
repetition of letters (i) is allowed, (ii) is not
allowed?
\item How many three-digit numbers can be
formed from the digits 0, 1, 3, 5, 6 if
repetitions of digits (i) are allowed, (ii) are
not allowed?
\item How many numbers between 100 and 1000
have 4 in the units place?
		 
\item Write in terms of factorials
(i)	$5 \times 6 \times 7 \times 8 \times 9 \times 10	$
(ii)	$3 \times 6 \times 9 \times 12 \times 15$
(iii)	$6 \times 7 \times 8 \times 9$
(iv)  $5 \times 10 \times 15 \times 20$

\item 
Evaluate : $\frac{n!}{r!(n-r)!}$ for  (i)	 n = 8, r = 6	 (ii)	 n = 12, r = 12,
	 (iii)	 n = 15, r = 10	 (iv)	 n = 15, r = 8
	 
\item Find n if
(i) $\frac{n}{8!}= \frac{3}{6!}+\frac{1!}{4!}$
(ii) $\frac{n}{6!}= \frac{4}{8!}+\frac{3}{6!}$
(iii) $\frac{1!}{n!}= \frac{1!}{4!}+\frac{4}{5!}$
(iv) $(n+1)!=42 \times (n-1)!$
(v) $(n+3)!=110 \times (n+1)!$

\item Find n if
(i) $\frac{(17-n)!}{(14-n)!}= 5!$
(ii) $\frac{(15-n)!}{(13-n)!}= 12$
(iii) $\frac{n!}{3!(n-3)!}:\frac{n!}{5!(n-5)!}=5:3$
(iv) $\frac{n!}{3!(n-3)!}:\frac{n!}{5!(n-7)!}=1:6$
(v) $\frac{(2n)!}{7!(2n-7)!}:\frac{n!}{4!(n-4)!}=21:1$ 

\item Find n if $^nP_6 : ^nP_3=120:1$

\item Find m and n, if $^{m+n}P_2 = 56 and ^{m+n}nP_2=12$

\item Find r if $^12P_{r-2} : ^11P_{r-1}=3:14$

\item Find the number of permutations of the
letters of the word UBUNTU.

\item How many 4 letter words can be formed
using letters in the word MADHURI if
(a) letters can be repeated (b) letters cannot
be repeated.



\item Determine the number of arrangements of
letters of the word ALGORITHM if
(a) vowels are always together.
(b) no two vowels are together.
(c) consonants are at even positions.

\item Find the number of arrangements of the letters
in the word SOLAPUR so that consonents
and vowels are placed alternately.

\item Find the value of  (a) $ ^{16}C_4 $
(b)  $ ^{80}C_2$
(c)  $ ^{15}C_4 + ^{15}C_5 $
(d)  $ ^{20}C_{16} - ^{19}C_{16} $

\item Find n if  (a) $ ^6P_2 = n ^6C_2 $
(b)  $ ^{2n}C_3 : ^nC_2 = 52:3$
(c)  $ ^nC_{n-3}=84$

\item Find the number of ways of selecting a team
of 3 boys and 2 girls from 6 boys and 4 girls.

\item Find r if  $ ^{11}C_4 + ^{11}C_5 + ^{12}C_6 +  ^{13}C_7=^{14}C_r$
 
 \item A group consists of 9 men and 6 women.
A team of 6 is to be selected. How many
of possible selections will have at least
3 women?
\end{enumerate} 




%////////////////////////////////////
\section{Method Of Indiction and B.T.}
\noindent
\begin{enumerate}[resume]


\item Prove by Induction : $1^2+3^2+5^2+...+(2n-1)^2 = \frac{n}{3}(2n-1)(2n+1)$ ii) $1\cdot2+2\cdot3+...+n(n+1) = \frac{n}{3}(n+1)(n+2)$ iii) $\frac{1}{1\cdot3}+\frac{1}{3\cdot5}+\frac{1}{5\cdot7}+...+\frac{1}{(2n-1)(2n+1)} = \frac{n}{2n+1}$ iv) $(2^{3n}-1)$ is divisible by 7

\item Find the value of i) $\left(\sqrt{3}+1\right)^4 - \left(\sqrt{3}-1\right)^4$ 
ii) $\left(2+\sqrt{5}\right)^5 - \left(2 -\sqrt{5}\right)^5$ iii) $\left(\sqrt{3}+\sqrt{2}\right)^6 + \left(\sqrt{3} -\sqrt{2}\right)^5$ 

\item Find the value of $(1.02)^6$, correct upto four places of decimals.

\item Find i) $9^{th}$ term of $\left(\frac{1}{3}+a^2\right)^2$ ii) coefficient of $x^8$ in $\left(2x^5 - \frac{5}{x^3}\right)^8$ iii) coefficient of $x^{-20}$ in $\left(x^3 - \frac{1}{2x^2}\right)^{15}$
iv) constant term in $\left(\sqrt{x} - \frac{3}{x^2}\right)^{10}$ v) constant term in $\left(x^2 - \frac{1}{x}\right)^9$
vi) middle terms of $\left(x^4 - \frac{1}{x^3}\right)^{15}$

\end{enumerate} 


\section{Set Theory}
\noindent
\begin{enumerate}[resume]

\item  If A = \{ 1, 2, 3, 4\}, B = \{3, 4, 5, 6\}
		  C = \{4, 5, 6, 7, 8\} and universal set  X = \{1, 2, 3, 4, 5, 6, 7, 8, 9, 10\}, then verify
the following:
i)		$ A \cup B\cap C) = (A\cup B) \cap (A \cup C)$

ii)	$A\cap (B\cup C)=(A\cap B)\cup (A \cup C)$
iii) $(A\cup B)' = (A'\cap B)'         $
iv)	$ (A\cap B)' = A'\cup B'          $
v)	$	 A = (A\cap B)\cup (A\cap B') $
vi)	$	 B = (A\cap B)\cup (A'\cap B) $
vii) $(A\cup B)=(A-B)\cup (A\cap B)\cup (B-A)$
viii)	$A \cap (B\Delta C) = (A\cap B) \Delta (A\cap C)$
ix) $n (A\cup B) = n(A) + n(B) - n(A\cap B)$ 
x) $n (B) = n(A'\cap B) + n(A\cap B)$

\item If A and B are subsets of the universal set X and n(X) = 50, n(A) = 35, n(B) = 20,$ n(A'\cap B') = 5,$ find i) $n (A\cup B)$ ii) $n(A\cap B)$
iv) $ n(A\cap B')$
iii) $ n(A'\cap B) $

\item In a class of 200 students who appeared
certain examinations, 35 students failed
in CET, 40 in NEET and 40 in JEE,
20 failed in CET and NEET, 17 in NEET
and JEE, 15 in CET and JEE and 5 failed
in all three examinations. Find how many
students,
i)	did not fail in any examination.
ii) failed in NEET or JEE entrance.
		 
		 
\item In a hostel, 25 students take tea, 20 students
take coffee, 15 students take milk, 10
student take bot tea and coffee, 8 students
take both milk and coffee. None of them
take tea and milk both and everyone takes
atleast one beverage, find the total number
of students in the hostel.

\item If $P = \{1, 2, 3 \} $ and $Q = \{14 \}$,find sets $P \times Q $ and $Q \times P$

\item Let A = \{1, 2, 3, 4\}, B = \{4, 5, 6\}, C = \{ 5, 6 \}.
 Verify, i) $A \times (B \cap C) = (A \times B) \cap (A \times C)$				
ii) $ A \times (B \cup C) = (A \times B) \cup (A \times C)$

\item Let A = \{6, 8\} and B = \{1, 3, 5\} Show that $ R_1 = \{ (a, b) / a \in A, b\in B, a - b$
is an even number \} is a null relation.
$R_2 = \{(a, b)/a \in A, b \in B$, a+b is odd number \}
is an universal relation.
\end{enumerate} 

END OF TERM ONE

%%/////////////////////////////////
\section{Circle}
\noindent
\begin{enumerate}[resume]
 \item Find the equation of the circle with
(i) Centre at origin and radius 4.
(ii) Centre at (-3, -2) and radius 6.
\item Find the centre and radius of the circle.
(i) $x^2 + y^2 = 25$ (ii) $\left(x -\frac{1}{2}\right)^2 + \left(y +\frac{1}{3}\right)^2 = \frac{1}{36}$
\item Find the equation of the circle if the equations of two diameters are $2x + y = 6$ and $3x + 2y = 4$ and the radius of circle is 9.
\item Find the equation of a circle passing through the origin and having intercepts 4 and -5 on
the co-ordinate axes.

\item Find the equation of a circle passing through the points (1,-4), (5, 2) and having its centre on the line $x-2y+9 =0$.

\item Find the centre and radius of the circle $x^2 + y^2 - 6x - 8y - 24 = 0$ 

\item Find the centre and radius of the circle $x^2 + y^2 - x +2y - 3 = 0$ 

\item Find the equation of the circle passing through the points (5, 7), (6, 6) and (2, -2).

\item If the circle passes through the points (0, 0), (a, 0) and (0, b), find the co-ordinates of its centre.

\end{enumerate} 


%////////////////////////////////////
\section{Mesures of Dispersion}
\noindent
\begin{enumerate}[resume]


\item Compute range for the following data:

\begin{tabular}{|c|*{11}{c|}}
\hline Classes & 62-64 & 64-66 & 66-68 & 68-70 & 70-72  \\
\hline Frequency  & 5 & 3 & 4 & 5 & 3 \\
\hline
\end{tabular}


\item Find variance and S.D. for the following set of numbers 65, 77, 81, 98, 100, 80, 129

\item Compute variance for the following data:

\begin{tabular}{|c|*{11}{c|}}
\hline Age in Years & 16 & 17 & 18 & 19 & 20 & 21  \\
\hline No. of Students  & 20 & 7 & 11 & 17 &30 & 15  \\
\hline
\end{tabular}

\item Compute variance and standard deviation for the following data:

\begin{tabular}{|c|*{11}{c|}}
\hline X & 2 & 4 & 6 & 8 & 10 & 12 & 14 & 16 & 18 & 20  \\
\hline F & 8 & 10 & 10 & 7 & 6 & 6 & 3 & 4 & 2 & 6 \\
\hline
\end{tabular}

\item Compute variance and standard deviation :

\begin{tabular}{|c|*{9}{c|}}
\hline X & 31 & 32 & 33 & 34 & 35 & 36 & 37 \\
\hline F & 15 & 12 & 10 & 8 & 9 & 10 & 6 \\
\hline
\end{tabular}

\item The means of two samples of sizes 60 and 120 respectively are 35.4 and 30.9 and the standard deviations 4 and 5. Obtain the standard deviation of the sample of size 180 obtained by combining the two sample.


\item For a certain data, following information is available.

\begin{tabular}{|c|*{9}{c|}}
\hline  & X & Y  \\
\hline Mean & 13 & 17  \\
\hline S.D. & 3 & 2  \\
\hline Size & 20 & 30  \\
\hline
\end{tabular}\\
\vspace{1mm}
Obtain the combined standard deviation.

\item A group of 65 students of class XI have their average height is 150.4 cm with coefficient of variance 2.5\%. What is the standard deviation of their height?


\item Given below is the information about marks obtained in Mathematics and Statistics by 100 students in a class. Which subject shows
the highest variability in marks?
\begin{tabular}{|c|*{9}{c|}}
\hline  & Mathematics & Statistics \\
\hline Mean & 20 & 25  \\
\hline S.D. & 2 & 3  \\
\hline
\end{tabular}\\

\end{enumerate} 


%%%////////////////////////////////////
\section{Functions}
\noindent
\begin{enumerate}[resume]


 \item  Which sets of ordered pairs represent
functions from A = \{1, 2, 3, 4\} to B = \{-1, 0,
1, 2, 3\}? Justify. 
	 (a)	 \{(1,0), (3,3), (2,-1), (4,1), (2,2) \}    
	 (b) 	\{(1,2), (2,-1), (3,1), (4,3) \}   	 (c) 	\{(1,3), (4,1), (2,2)\}	 (d) 	\{(1,1), (2,1), (3,1), (4,1)\}
 
 \item Find $x$, if $g (x) = 0$ where  (a) $g(x)=\frac{5x-6}{7}$ (b) $g(x)=\frac{18-2x^2-6}{7}$ 
 
 \item Find $x$, if $g^2 (x) = 0$ where  $g(x)=\frac{18-2x^2}{7}$ 
 

\item Express the following exponential equations
in logarithmic form  (a) $2^5=32$ (b) $9^\frac{3}{2}=27$ (c) $3^{-4} = \frac{1}{81}$

\item Express the following logarithmic equations 	
in exponential form (a) $\log_5\left( \frac{1}{25}\right)=-2$ (b) $\log_\frac{1}{2}(-8)=3$ 

\item Write $5\log x + 7\log y -\log z$ as a single
logarithm.

\item Solve for $x$ (a) $\log2 +\log(x+3) -\log(3x-5) =\log3$ (b) $x +\log_{10} (1+2^x ) = x\log_{10} 5 +\log_{ 10}6$

\item If  $\log\left(\frac{x+y}{3}\right)=\frac{1}{2}\log x+ \frac{1}{2}\log y, $ show that $\frac{x}{y}+\frac{y}{x}=7$

\item If  $\log\left(\frac{x-y}{4}\right)= \log \sqrt{x}+ \log \sqrt{y}, $ show that $(x+y)^2=20xy$

\item If  $f (x) = 2x^2 + 3, g (x) = 5x - 2, $then find
	(a) f ◦ g
	(b) f ◦ f	

\item Verify that $ f $  and  $ g $ are inverse functions of each other, where (a) $ f(x) = \frac { x - 7 }{4}, g(x) = 4x + 7 $ (b) $ f(x) = x^3 + 4, g(x) = \sqrt[3]{x - 4 } $

\item $ f(x) = 2 \left| x \right| + 3x $ then find (a) $ f(2)$ (b) $ f(-5) $

\end{enumerate} 

%%%////////////////////////////////////
\section{Limits}
\noindent
\begin{enumerate}[resume]
\item Evaluate : \raisebox{0.5ex}{$ \lim\limits_{x \to 2} \frac{ x^{-3} - 2^{-3}}{x-2} $}
 
\item If $ \lim\limits_{x \to 2} \frac{ x^4 - 1}{ x - 1} = \lim\limits_{x \to 2} \frac{ x^3 - a^3}{x-a} $ find a.

\item Evaluate : $ \lim\limits_{x \to 0} \frac{ (1- x)^8 - 1}{(1- x)^2 - 1} $ 

 \item Evaluate : $ \lim\limits_{ x \to 1 } \frac{ x + x^3 + x^5 +\cdots +x^{2n - 1} - n }{x-1} $
 
 \item Evaluate : $ \lim\limits_{x \to 3} \frac{ x + 3}{ x^2 +4x + 3 } $
 
\item Evaluate : $ \lim\limits_{x \to 3} \frac{ x^2 + 2x - 15}{ x^2 - 5x + 6 } $

\item Evaluate : $ \lim\limits_{x \to 2} \frac{ x^3 - 7x + 6}{ x^3 - 7x^2 + 16x - 12   } $

\item Evaluate : $ \lim\limits_{ x \to 1 } \left[ \frac{ x + 2 }{x^2-5x + 4} + \frac{ x - 4 }{3(x^2-3x+2)} \right] $

\item Evaluate : $ \lim\limits_{x \to 3} \frac{ \sqrt{ 2x + 3 } - \sqrt{ 4x - 3 } }{ x^2 - 9 } $

\item Evaluate : $ \lim\limits_{x \to 2} \frac{ x^2 - 4 }{ \sqrt{ x + 2 } - \sqrt{ 3x - 2 }  } $

 \item Evaluate : $ \lim\limits_{ \theta \to 0} \frac{ \sin m\theta }{ \tan n\theta}$
 
 \item Evaluate : $ \lim\limits_{ \theta \to 0} \frac{ 1 -  \cos 2\theta }{ \theta ^2  } $
 
\item Evaluate : $ \lim\limits_{ x \to \frac {\pi}{3} } \frac{ 2 -  \operatorname{cosec} \ x }{ \cot^2 x - 3  } $
 
\item Evaluate : $ \lim\limits_{ x \to \pi } \frac{ \sqrt{ 5 + \cos x } -2 }{ ( \pi - x )^2  } $
 
\item Evaluate : $ \lim\limits_{ x \to 1 } \frac{ 1 - x^2 }{ \sin \pi x  } $
 
 
 
 \item Evaluate : $ \lim\limits_{ x \to 0 } \frac{ 5^x - 3^x - 2^x - 1 }{ x} $
 
 \item Evaluate : $ \lim\limits_{ x \to 0 } \frac{ 6^x + 5^x + 4^x - 3^{x+1} }{ x} $
 
 \item Evaluate : $ \lim\limits_{ x \to 0 } \left[ \frac{ 3+x }{3-x}\right]^{\frac{1}{x}} $
 

  \item Evaluate : $ \lim\limits_{ x \to 0 } \frac{ \log (3-x) - \log (3+x) }{ x} $
  
  
   \item Evaluate : $ \lim\limits_{ x \to 0 } \frac{ \left(2^x - 1\right)^3 }{ (3^x - 1)\; \cdot \;\sin x \; \cdot \; \log (1+x)} $
  
  
 \item Evaluate : $ \lim\limits_{ x \to 0 } \frac{ (25)^x - 2(5)^x +1}{ x\sin x} $
 \item Evaluate : $ \lim\limits_{ x \to \infty} \frac{x^3+ 3x+2}{(x+4)(x-6)(x-3)} $

 \item Evaluate : $ \lim\limits_{ x \to \infty } \frac{ a^3 + bx^2+cx +d}{ ex^3+fx^2+gx+h} $

 \item Evaluate : $ \lim\limits_{ x \to \infty} \frac{x^3+ 3x+2}{(x+4)(x-6)(x-3)} $

\item Evaluate : $ \lim\limits_{ x \to \infty } \sqrt{x^2+4x+16} - \sqrt{ x^2+16} $

\item Evaluate : $ \lim\limits_{ x \to \infty} \frac{(3x^2+4)(4x^2-6)(5x^2+2)}{4x^6+2x^4-1} $

\item Evaluate : $ \lim\limits_{ x \to \infty }\sqrt{x}\left( \sqrt{x+1} - \sqrt{x}\right) $

\end{enumerate} 





%%%////////////////////////////////////
\section{Conic Sections}
\noindent
\begin{enumerate}[resume]
 \item  Find co-ordinate of focus, equation of directrix, length of latus rectum and the co ordinate of end points of latus rectum of the parabola i) $5y^2=24x\ $ ii) $y^2 = –20x$ iii) $3x^2 = 8y$ iv) $x^2 = –8y$ v) $3y ^2 = –16x$
\item Find the equation of the parabola with vertex at the origin, axis along X-axis and passing through the point (3,4).
\item For the parabola $3y^2 =16x$, find the parameter of the point a) (3,–4) b) (27,–12).

\item Find coordinate of the point on the parabola. Also find focal distance. i) $y 2 = 12x$ whose parameter is $\frac{1}{3}$.
 
\item Find length of latus rectum of the parabola $y^2 = 4ax$ passing through the point (2.–6).

\item Find the (i) lengths of the principal axes. (ii) co-ordinates of the focii (iii) equations of directrics (iv) length of the latus rectum (v) distance between focii (vi) distance between directrices of the ellipse: 
(a) $\frac{x^2}{25} + \frac{y^2}{9} = 1$ (b) $3x^2 + 4y^2 = 12$

\item Find the equation of the 		 ellipse in standard form if
i) eccentricity = $\frac{3}{8}$ and distance between its focii = 6.
(ii) distance between directrix is 18 and eccentricity is $\frac{1}{3}$

\item Find the eccentricity of an ellipse, if the length of its latus rectum is one third of its minor axis.

\item Show that the line $x – y = 5$ is a tangent to the ellipse $9x^2 + 16y^2 = 144$. Find the point of contact.

\item Find the equation of the tangent to the ellipse (i) $\frac{x^2}{5} + \frac{y^2}{4} = 1$ passing through the point (2, -2). iii) $2x^2 + y^2$ = 6 from the point (2, 1).

\item Find the length of transverse axis, length of conjugate axis, the eccentricity, the co-ordi-nates of foci, equations of directrices and the
length of latus rectum of the hyperbola. i) $\frac{x^2}{25}-\frac{y^2}{16} = 1$ ii) $16x^2 - 9y^2 = 144$

\item Find the eccentricity of the hyperbola, which is conjugate to the hyperbola $x^2 - 3y^2 = 3$

\item Find the equation of the hyperbola referred to its principal axes. i) whose distance between foci is 10 and eccentricity $\frac{5}{2}$
ii) whose distance between foci is 10 and length of conjugate axis 6.

\end{enumerate} 





%%%////////////////////////////////////
\section{Probability}
\noindent
\begin{enumerate}[resume]
 \item There are four pens: Red, Green, Blue and
Purple in a desk drawer of which two pens
are selected at random one after the other
with replacement. State the sample space
and the following events. a)	 A : Selecting at least one red pen. b)	 B : Two pens of the same color are not
selected.

\item A coin and a die are tossed simultaneously.
Enumerate the sample space and the
following events. a)	 A : Getting a Tail and an Odd number b)	 B : Getting a prime number
	
\item	 Find n(S) for each of the following random
experiments.  a) From an urn containing 5 gold and 3
silver coins, 3 coins are drawn at random b)	 5 letters are to be placed into 5 envelopes
such that no envelop is empty.

\item A fair die is thrown two times. Find the
probability that a)	 sum of the numbers on them is 5
b)	 first throw gives a multiple of 2 and
second throw gives a multiple of 3.
\item Two cards are drawn from a pack of 52
cards. Find the probability that a)	 one is a face card and the other is an ace card
b)	 one is club and the other is a diamond
c)	 both are from the same suit.
\item From a bag containing 10 red, 4 blue and
6 black balls, a ball is drawn at random. Find
the probability of drawing
a)	 a red ball. b)	 a blue or black ball.

\item A room has three sockets for lamps. From a
collection 10 bulbs of which 6 are defective.
At night a person selects 3 bulbs, at random
and puts them in sockets. What is the
probability that i) room is still dark ii) the
room is lit

\item A card is drawn from a pack of 52 cards.
What is the probability that,
i)	 card is either red or black?	
ii)	 card is either black or a face card?

\item Form a group of 4 men, 4 women and
3 children, 4 persons are selected at random.
Find the probability that, i) no child is
selected ii) exactly 2 men are selected.

\item For two events A and B of a sample space S. If P(A) = $\frac{3}{8}$, P(B) = $\frac{1}{2}$ and $P(A\cup B)=\frac{5}{8}$, find the value of 
 i) $P (A\cap B)$ ii) $P(A' \cap B')$
iii) $ P(A' \cup B')$

\item A bag contains 3 red marbles and 4 blue
marbles. Two marbles are drawn at random
without replacement. If the first marble
drawn is red, what is the probability the
second marble is blue?

\item From a pack of well-shuffled cards, two cards
are drawn at random. Find the probability
that both the cards are diamonds when i)	 first card drawn is kept aside  ii)	 the first card drawn is replaced in the pack.

\item A, B, and C try to hit a target simultaneously
but independently. Their respective
probabilities of hitting the target are  $\frac{3}{4}$, $\frac{1}{2}$ and $\frac{5}{8}$ and . Find the probability that the target a) is hit exactly by one of them b) is not hit by any one of them
c) is hit d) is exactly hit by two of them.

\item 	Three fair coins are tossed. What is the
probability of getting three heads given that
at least two coins show heads?

\item There are three bags, each containing 100
marbles. Bag 1 has 75 red and 25 blue
marbles. Bag 2 has 60 red and 40 blue marbles
and Bag 3 has 45 red and 55 blue marbles.
One of the bags is chosen at random and a
marble is picked from the chosen bag. What
is the probability that the chosen marble is
red?

\item There is a working women's hostel in a
town, where 75\% are from neighbouring
town. The rest all are from the same town.
48\% of women who hail from the same
town are graduates and 83\% of the women
who have come from the neighboring town
are also graduates. Find the probability that
a woman selected at random is a graduate
from the same town.

\end{enumerate} 

%%%////////////////////////////////////
\section{Continuity}
\noindent
\begin{enumerate}[resume]


\item Examine the continuity of
$f(x) =  \begin{cases} 
       \frac{x^2 - 9}{x -3} & x\neq 3 \\
        8                            &  x = 3 
   \end{cases}$


\item Examine the continuity of
$f(x) =  \begin{cases} 
       \frac{x^2 - 8x - 19}{x-1} & x\neq 1 \\
        20                           &  x = 1 
   \end{cases}$
   
   
  \item Examine the continuity of
$f(x) =  \begin{cases} 
       \frac{x^3 - 8}{\sqrt{x+2} - \sqrt{3x-2}} & x\neq 2 \\
        -24                            &  x = 2 
   \end{cases}$

\item Examine the continuity of
$f(x) =  \begin{cases} 
        x^2+3x-2 & x\leq 4 \\
        5x+3                           &  x > 4
   \end{cases}$

\item Examine the continuity of
$f(x) =  \begin{cases} 
        4+\sin x  & x < \pi \\
        3 - \cos x   &  x > \pi 
   \end{cases}$
   
   
   \item Examine continuity of
$f(x) =  \begin{cases}  
       \frac{\sqrt{3} - \tan x}{\pi-3x} & x\neq \frac{\pi}{3} \\
        \frac{3}{4}                       &  x = \frac{\pi}{3}
   \end{cases}$ at $x = \frac{\pi}{3}$

\item Examine continuity of
$f(x) =  \begin{cases} 
       \frac{4^x - 2^{x+1}+1}{1- \cos 2x} & x\neq 0 \\ 
        \frac{(\log 2)^2}{2}                           &  x = 0
   \end{cases}$ at $ x  = 0$

\item If
$f(x) = 
       \frac{\sqrt{2+\sin x}- \sqrt{3}}{\cos^2x}$, for $x \neq \frac{\pi}{2}$, is continuous at $x = \frac{\pi}{2}$ the find $f(\frac{\pi}{2}) $


\item If
$f(x) =  \frac{5^x+5^{-x}-2}{\cos^2x}$, for $x \neq \frac{\pi}{2}$, is continuous at $x = \frac{\pi}{2}$ the find $f(\frac{\pi}{2}) $

\item If
$f(x) =  \begin{cases} 
       \frac{5^x+5^{-x}-2}{x^2} & x\neq 0 \\
        k                            &  x = 0 
   \end{cases}$\, is continuous at x=0, find k.


\item If
$f(x) =  \begin{cases} 
       \frac{\sin 2x}{5x} - a& x>0 \\
        4                           &  x = 0 \\
        x^2+b-3                &  x = 0 
   \end{cases}$\, is continuous at x=0, find a and b.
  
  \end{enumerate} 
   
   %%%////////////////////////////////////
\section{Differentiation}
\noindent
\begin{enumerate}[resume]


\item Find derivative w.r.t $x$using first principle (a) $x^2+3x-1$ (b) $\sin (3x)$ (c) $e^{2x+1}$ 

\item Find the derivatives of the following w. r. t. x. at the points indicated against them by using method of first principle (a) $\sqrt{2z+5}$ at $e) x=2$ (b) $\tan x$ at $x = \frac{\pi}{4}$ (c) $e^{3x-4}$ at $x=2$

\item Show that the function $f(x) =  \begin{cases} 
        x^2+2 & x \leq -3 \\
        2-3x    &  x \geq -3 
   \end{cases}$ is not differentiable at $x=-3$

\item Differentiate w.r.t. x (a) $y=x^\frac{4}{3}+e^x-\sin x$ (b) $y=\log x - cosec x + 5^x - \frac{3}{x^\frac{3}{2}}$ (c) $y=7^x+x^7-\frac{2}{3}x\sqrt{x} = \log x + 7^7$ (d) $y=x^5\tan x $ (e) $y = e^x \log x$ (f) $y=x^2\sqrt{x} + x^4\log x$ (g) $y=\frac{x^2+3}{x^2-5}$ (h) $y=\frac{xe^x}{x+e^x}$

\item  If $f(x)$ is a quadratic polynomial such that $f(0) = 3, f '(2) = 2$ and $f '(3) = 12$
then find $f(x)$.

\item  If $f(x)=a\sin x - b\cos x , f'\left(\frac{\pi}{4}\right)=\sqrt{2}$ and $f'\left(\frac{\pi}{6}\right)=2$
then find $f(x)$.

\end{enumerate} 


\end{multicols}
 
\end{document}