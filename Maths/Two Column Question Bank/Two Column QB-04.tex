\documentclass[14pt]{article}
\usepackage[a4paper, total={7in, 8in}]{geometry}
\usepackage[utf8]{inputenc}
\usepackage[english]{babel}
\usepackage{paralist} 
\usepackage{multicol}
\usepackage{amsmath}
\usepackage{amssymb}
\usepackage{nccmath}
\setlength{\columnsep}{1cm}
 
\begin{document}
\begin{multicols}{2}
[
\section{Set Theory}
All human things are subject to decay. And when fate summons, Monarchs must obey.
]
\noindent
 \begin{inparaenum}[(1)]
    \item $ A\cap B$
     \item Subitem one two.
     \item Subitem one three.
   \end{inparaenum}

\begin{enumerate}
  \item The numbers starts at 1 with every call to the enumerate environment.
Hello, here is some text without a meaning.  This text should show what 
a printed text will look like at this place.
If you read this text, you will get no information.  Really?  Is there 
no information?  Is there...
Hello, here is some text without a meaning.  This text should show what 
a printed text will look like at this place.
If you read this text, you will get no information.  Really?  Is there 
no information? A teacher has 2
I English, 3 diilerent books on Physics, and
v 4 diilerent books on Mathematics. These books
are to be placed in a shelfso that all books on any
1 one subjects are together. How many diilerent
g ‘ ways are there to do this? sf
\end{enumerate}
  \begin{inparaenum}[(1)]
     \item Subitem one one.
     \item Subitem one two.
     \item Subitem one three.
   \end{inparaenum}
\end{multicols}
 
\end{document}