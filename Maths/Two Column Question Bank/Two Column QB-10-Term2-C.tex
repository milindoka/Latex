%\documentclass[14pt]{article}
\documentclass[17pt]{extarticle}
\usepackage[a4paper, total={7.8in, 10.5in}]{geometry}
\usepackage[utf8]{inputenc}
\usepackage[english]{babel}
\usepackage{paralist} 
\usepackage{multicol}
\usepackage{amsmath}
\usepackage{amssymb}
\usepackage{enumitem}
\usepackage{textcomp}


\newcommand{\degree}{$^{\circ}\ $} %added space after degree

\setlength{\columnsep}{1mm}

\begin{document}
\centering 
{\huge \bf Mathematics QB From 2019-20 onwards\par}
\vspace{1cm}
\begin{multicols}{2}
%%/////////////////////////////////
\section{Circle}
\noindent
\begin{enumerate}
 \item Find the equation of the circle with
(i) Centre at origin and radius 4.
(ii) Centre at (-3, -2) and radius 6.
\item Find the centre and radius of the circle.
(i) $x^2 + y^2 = 25$ (ii) $\left(x -\frac{1}{2}\right)^2 + \left(y +\frac{1}{3}\right)^2 = \frac{1}{36}$
\item Find the equation of the circle if the equations of two diameters are $2x + y = 6$ and $3x + 2y = 4$ and the radius of circle is 9.
\item Find the equation of a circle passing through the origin and having intercepts 4 and -5 on
the co-ordinate axes.

\item Find the equation of a circle passing through the points (1,-4), (5, 2) and having its centre on the line $x-2y+9 =0$.

\item Find the centre and radius of the circle $x^2 + y^2 - 6x - 8y - 24 = 0$ 

\item Find the equation of the circle passing through the points (5, 7), (6, 6) and (2, -2).

\end{enumerate} 


%%%////////////////////////////////////
\section{Conic Sections}
\noindent
\begin{enumerate}
 \item  Find co-ordinate of focus, equation of directrix, length of latus rectum and the co ordinate of end points of latus rectum of the parabola i) $5y^2=24x\ $ ii) $y^2 = –20x$ iii) $3x^2 = 8y$ iv) $x^2 = –8y$ v) $3y ^2 = –16x$
\item Find the equation of the parabola with vertex at the origin, axis along X-axis and passing through the point (3,4).
\item For the parabola $3y^2 =16x$, find the parameter of the point a) (3,–4) b) (27,–12).

\item Find coordinate of the point on the parabola. Also find focal distance. i) $y 2 = 12x$ whose parameter is $\frac{1}{3}$.
 
\item Find length of latus rectum of the parabola $y^2 = 4ax$ passing through the point (2.–6).

\item Find the (i) lengths of the principal axes. (ii) co-ordinates of the focii (iii) equations of directrics (iv) length of the latus rectum (v) distance between focii (vi) distance between directrices of the ellipse: 
(a) $\frac{x^2}{25} + \frac{y^2}{9} = 1$ (b) $3x^2 + 4y^2 = 12$

\item Find the equation of the 		 ellipse in standard form if
i) eccentricity = $\frac{3}{8}$ and distance between its focii = 6.
(ii) distance between directrix is 18 and eccentricity is $\frac{1}{3}$

\item Find the eccentricity of an ellipse, if the length of its latus rectum is one third of its minor axis.

\item Show that the line $x – y = 5$ is a tangent to the ellipse $9x^2 + 16y^2 = 144$. Find the point of contact.

\item Find the equation of the tangent to the ellipse (i) $\frac{x^2}{5} + \frac{y^2}{4} = 1$ passing through the point (2, -2). iii) $2x^2 + y^2$ = 6 from the point (2, 1).

\item Find the length of transverse axis, length of conjugate axis, the eccentricity, the co-ordi-nates of foci, equations of directrices and the
length of latus rectum of the hyperbola. i) $\frac{x^2}{25}-\frac{y^2}{16} = 1$ ii) $16x^2 - 9y^2 = 144$

\item Find the eccentricity of the hyperbola, which is conjugate to the hyperbola $x^2 - 3y^2 = 3$

\item Find the equation of the hyperbola referred to its principal axes. i) whose distance between foci is 10 and eccentricity $\frac{5}{2}$
ii) whose distance between foci is 10 and length of conjugate axis 6.

\end{enumerate} 

%////////////////////////////////////
\section{Mesures of Dispersion}
\noindent
\begin{enumerate}
\item Find variance and S.D. for the following set of numbers 65, 77, 81, 98, 100, 80, 129

\item Compute variance and standard deviation for the following data:

\begin{tabular}{|c|*{11}{c|}}
\hline X & 2 & 4 & 6 & 8 & 10 & 12 & 14 & 16 & 18 & 20  \\
\hline F & 8 & 10 & 10 & 7 & 6 & 6 & 3 & 4 & 2 & 6 \\
\hline
\end{tabular}

\item Compute variance and standard deviation :

\begin{tabular}{|c|*{9}{c|}}
\hline X & 31 & 32 & 33 & 34 & 35 & 36 & 37 \\
\hline F & 15 & 12 & 10 & 8 & 9 & 10 & 6 \\
\hline
\end{tabular}

\item The means of two samples of sizes 60 and 120 respectively are 35.4 and 30.9 and the standard deviations 4 and 5. Obtain the standard deviation of the sample of size 180 obtained by combining the two sample.


\item For a certain data, following information is available.

\begin{tabular}{|c|*{9}{c|}}
\hline  & X & Y  \\
\hline Mean & 13 & 17  \\
\hline S.D. & 3 & 2  \\
\hline Size & 20 & 30  \\
\hline
\end{tabular}\\
\vspace{1mm}
Obtain the combined standard deviation.

\item A group of 65 students of class XI have their average height is 150.4 cm with coefficient of variance 2.5\%. What is the standard deviation of their height?

\item Given below is the information about marks obtained in Mathematics and Statistics by 100 students in a class. Which subject shows
the highest variability in marks?
\begin{tabular}{|c|*{9}{c|}}
\hline  & Mathematics & Statistics \\
\hline Mean & 20 & 25  \\
\hline S.D. & 2 & 3  \\
\hline
\end{tabular}\\






\end{enumerate} 



\end{multicols}
 
\end{document}