\documentclass[17pt]{extarticle}
\usepackage{amsmath, amssymb}
\usepackage{nccmath}
\usepackage[a4paper, total={6.5in, 10.5in},top=5mm,left=27mm]{geometry}
\usepackage{titlesec}
\usepackage{tikz}
\usepackage{gensymb}


\titleformat{\section}
{\normalfont\normalsize\bfseries}{\thesection}{1em}{}

\begin{document}
\noindent
\begin{fleqn} 

%----------------------------------------

\section{Question: 01}
Find the equation fo the pair of lines through the origin and making an equilateral triangle with the line $y = 3.$

%----------------------------------------

\section{Answer: 01}
Let OA and OB be the lines passing through the origin making an angle of $ 60\degree$ with the line $y = 3.$ \\
%\vrule
$\therefore$ OA and OB make an angle of $60\degree$ and $ 120\degree$ with the positive direction of X-axis.\\ 
Slope of line OA = tan $60\degree$ = $\sqrt{3}$ \\
$\therefore$ Equation of the line OA is\\  
$y=$ $\sqrt{3}\ x$ \ i.e. $\sqrt{3}\ x - y = 0 $ \\ 
Slope of line OB = tan $120\degree$ \\
= tan $(180 - 60)\degree$ = -tan $60\degree$\\
$\therefore$ Slope of line OB = -$\sqrt{3}$ \\
$\therefore$ Equation of the line OB is\\  
$y=$ -$\sqrt{3}\ x$ \ i.e. $\sqrt{3}\ x + y = 0 $ \\
$\therefore$ The required joint equation of\\ the lines is\\
$\left( \sqrt{3}\ x - y\right)$ $\left( \sqrt{3}\ x + y\right) = 0 $\\
i.e.\ $ 3x^2 - y^2 = 0.$

\begin{equation} \nonumber
%\vrule
\quad
\begin{alignedat}{4}
\begin{tikzpicture}
\draw (-3,0) -- (3,0);
\draw (0,-3) -- (0,3);
\draw (-3,2) -- (3,2);
\draw (-1.73,-3) -- (1.73,3);
\draw (-1.73,3) -- (1.73,-3);
%\draw (1,0.5) -- (7,0.5); 
%\draw (4,-1) -- (4,4);
%\draw (0.5,3) -- (8,3); 
%\draw (3,-0.65) -- (7,4); 
%\draw (5,-0.65) -- (0,5);

\end{tikzpicture}
\end{alignedat}
\end{equation}
\quad
\begin{equation} \nonumber
\end{equation}
%----------------------------------------

\section{Question: 02}

Find the joint equation of lines through the origin each of which is makes an angle of $60\degree$ with the Y-axis.

%----------------------------------------


\section{Answer: 02}

%----------------------------------------


\end{fleqn}
\end{document} 