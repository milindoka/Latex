\documentclass[17pt]{extarticle}
\usepackage{amsmath, amssymb}
\usepackage{nccmath}
\usepackage[a4paper, total={6.5in, 10.5in},top=5mm,left=27mm]{geometry}
\usepackage{titlesec}
\usepackage{tikz}
\usepackage{gensymb}
\usetikzlibrary{arrows.meta}

\titleformat{\section}
{\normalfont\normalsize\bfseries}{\thesection}{1em}{}
\setcounter{secnumdepth}{0} %% no numbering for sections

\begin{document}
\noindent
\begin{fleqn} 

%%%%%%%%%%%%%%%%%%%%%%%%%%%%%%%%%%%%%%%%%%%%%%%%%%%%%%%%%%%%%%%%

\section{Question: 01}
Show that the difference between the slopes of the lines given by line $\left(\text{tan}^2\theta \ + \text{cos}^2\theta \right)$ $x^2$ $-$ 2$xy$ tan $\theta$ + $\left(\text{sin}^2\theta\right) y^2 $ = 0  is two.
 
%----------------------------------------

\section{Answer: 01}
Comparing, $\left(\text{tan}^2\theta \ + \text{cos}^2\theta \right)$ $x^2$ $-$ 2$xy$ tan $\theta$ + $\left(\text{sin}^2\theta\right) y^2 $ = 0 with $ax^2 + 2hxy +  by^2 = 0 $  \ we get, $a = \left(\text{tan}^2\theta \ + \text{cos}^2\theta \right),\\ 2h = - 2\ \text {tan}\theta \  \text{and}\ b = \text{sin}^2\theta $
\begin{equation} \nonumber
\begin{alignedat}{4}
& \text {Let} \ m_1 \ \text{and}\ m_2\ \text {be the slopes of the}\\ 
& \text {lines represented by given eqation}  \\
& \therefore m_1 + m_2 = \frac{-2h}{b} \\ 
& = -\left[\frac{-2\ \text {tan}\ \theta }{\text {sin}^2\ \theta } \right] \\
& = \frac{2\ \text{tan}\ \theta}{\text {sin}^2\ \theta}\\  
& y= \sqrt{3}\ x \ i.e. \sqrt{3}\ x - y = 0 \\ 
& \text {Slope of line OB = tan 120\degree} \\
& \text {= tan (180 - 60)\degree} = \text{-tan}\ 60\degree\\
& \therefore \text {Slope of line OB} = -\sqrt{3} \\
& \therefore \text{Equation of the line OB is}\\  
\end{alignedat}
\quad
\vrule
\quad
\quad\quad
\begin{alignedat}{4}
\text {= tan (180 - 60)\degree} = \text{-tan}\ 60\degree\\
\therefore \text {Slope of line OB} = -\sqrt{3} \\
\therefore \text{Equation of the line OB is}\\  
\end{alignedat}
\end{equation}
\quad

%%%%%%%%%%%%%%%%%%%%%%%%%%%%%%%%%%%%%%%%%%%%%%%%%%%%%%%%%%%%%%%%%%%%%%%%%%

\end{fleqn}
\end{document} 
