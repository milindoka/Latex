\documentclass[17pt]{extarticle}
\usepackage{amsmath, amssymb}
\usepackage{nccmath}
\usepackage[a4paper, total={8in, 11.38in},top=2mm,left=27mm,bottom=2mm,right=5mm]{geometry}
\usepackage{tikz}
\usepackage{titlesec}

\titleformat{\section}
{\normalfont\normalsize\bfseries}{\thesection}{1em}{}
\titleformat{\subsection}
{\normalfont\normalsize\bfseries}{\thesection}{1em}{}

\begin{document}

\noindent
\begin{fleqn} 

%%%%%%%%%%%%%%%%%%%%%%%%%%%%%%%%%%%%%%%%%%%%%%%%%%%%%%%%%%%%%%%%

\section{Question} 

$\begin{aligned}[t] 
\text{If \ } x + \sqrt{xy} + y = 1 \text{\ , \  find \ \  } \frac{dy}{dx}
\end{aligned}$

%----------------------------------------
\subsection*{Answer}
\begin{equation} \nonumber
\begin{alignedat}{4}
& \quad  \int_{0}^{\frac{\pi}{2}} \frac{\sin^2x}{(1 + \cos x)^2}\ dx\\
&= \int_{0}^{\frac{\pi}{2}} \frac{1 -\cos^2x}{(1 + \cos x)^2}\ dx\\
&= \int_{0}^{\frac{\pi}{2}} \frac{(1 -\cos x)(1 +\cos x)}{(1 + \cos x)^2}\ dx\\
&= \int_{0}^{\frac{\pi}{2}} \frac{1 -\cos x}{1 + \cos x}\ dx\\
&= \int_{0}^{\frac{\pi}{2}} \frac{2\sin^2\frac{x}{2}}{2\cos^2\frac{x}{2}}\ dx\\
\end{alignedat}
\quad
\vrule
\quad
\begin{alignedat}{4}
&= \int_{0}^{\frac{\pi}{2}} \tan^2\frac{x}{2}\ dx\\
&= \int_{0}^{\frac{\pi}{2}} \left(\sec^2\frac{x}{2}-1\right)dx\\
&= \left[2\tan\frac{x}{2}-x\right]_{0}^{\frac{\pi}{2}}\\
&= \left[2\tan\frac{\pi}{4}-\frac{\pi}{2}\right]-\left[2\tan 0 - 0 \right]\\
&= \left[2 -\frac{\pi}{2}\right]-\left[0 - 0 \right]\\
&= 2 -\frac{\pi}{2}
\end{alignedat}
\end{equation}
%%%%%%%%%%%%%%%%%%%%%%%%%%%%%%%%%%%%%%%%%%%%%%%%%%%%%%%%%%%%%%%%

\section{Question} 

$\begin{aligned}[t] 
\text{If \ } \int_{-\frac{\pi}{4}}^{\frac{\pi}{3}} \frac{1}{1 - \sin x}\ dx
\end{aligned}$

%----------------------------------------
\subsection*{Answer}
\begin{equation} \nonumber
\begin{alignedat}{4}
& \quad  \int_{-\frac{\pi}{4}}^{\frac{\pi}{3}} \frac{1}{1 - \sin x}\ dx\\
&= \int_{-\frac{\pi}{4}}^{\frac{\pi}{3}} \frac{1+\sin x}{(1 - \sin x)(1 + \sin x)}\ dx\\
&= \int_{-\frac{\pi}{4}}^{\frac{\pi}{3}} \frac{1+\sin x}{1 - \sin^2x}\ dx\\
&= \int_{-\frac{\pi}{4}}^{\frac{\pi}{3}} \frac{1+\sin x}{\cos^2x}\ dx\\
\end{alignedat}
\quad
\vrule
\quad
\begin{alignedat}{4}
&= \int_{-\frac{\pi}{4}}^{\frac{\pi}{3}} \left[\frac{1}{\cos^2x} + \frac{\sin x}{\cos^2x}\right] \ dx\\
&= \int_{-\frac{\pi}{4}}^{\frac{\pi}{3}} \left[\sec^2x + \sec x \tan x \right] \ dx\\
&= \left[\tan x + \sec x \right]_{-\frac{\pi}{4}}^{\frac{\pi}{3}}\\
&= \left[ \sqrt{3}+2\right] - \left[-1 + \sqrt{2}\right]\\
&= \sqrt{3}+3 - \sqrt{2} \\
\end{alignedat}
\end{equation}

%%%%%%%%%%%%%%%%%%%%%%%%%%%%%%%%%%%%%%%%%%%%%%%%%%%%%%%%%%%%%%%%

\end{fleqn}
\end{document} 