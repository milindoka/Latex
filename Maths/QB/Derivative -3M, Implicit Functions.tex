\documentclass[17pt]{extarticle}
\usepackage{amsmath, amssymb}
\usepackage{nccmath}
\usepackage{cancel}
\usepackage[a4paper, total={8in, 11.38in},top=2mm,left=27mm,bottom=2mm,right=5mm]{geometry}
\usepackage{tikz}
\usepackage{titlesec}

\titleformat{\section}
{\normalfont\normalsize\bfseries}{\thesection}{1em}{}
\titleformat{\subsection}
{\normalfont\normalsize\bfseries}{\thesection}{1em}{}

\begin{document}

\noindent
\begin{fleqn} 

%%%%%%%%%%%%%%%%%%%%%%%%%%%%%%%%%%%%%%%%%%%%%%%%%%%%%%%%%%%%%%%%

\section{Question} 

$\begin{aligned}[t] 
\text{If \ } x + \sqrt{xy} + y = 1 \text{\ , \  find \ \  } \frac{dy}{dx}
\end{aligned}$

%----------------------------------------
\subsection*{Answer}
\begin{equation} \nonumber
\begin{alignedat}{4}
& \text{Given \ } x + \sqrt{xy} + y = 1 \\
& \text{Differentiating w.r.t. x \ }\\
& 1 + \frac{1}{2\sqrt{xy}} \left( x\frac{dy}{dx}+y \right)+\frac{dy}{dx}=0 \\
& \therefore 1 + \frac{x}{2\sqrt{xy}} \frac{dy}{dx}+ \frac{y}{2\sqrt{xy}}+\frac{dy}{dx}=0 \\
& \therefore \left(\frac{x}{2\sqrt{xy}}+1\right) \frac{dy}{dx}=-1-\frac{y}{2\sqrt{xy}}\\
\end{alignedat}
\,
\vrule
\, 
\begin{alignedat}{4}
& \therefore \left(\frac{x+2\sqrt{xy}}{2\sqrt{xy}}\right) \frac{dy}{dx}=-\frac{2\sqrt{xy}+y}{2\sqrt{xy}}\\
& \therefore  (x+2\sqrt{xy}\,)\ \frac{dy}{dx}= -2\sqrt{xy}-y \\
& \therefore \frac{dy}{dx} =-\,\frac{x+2\sqrt{xy}}{2\sqrt{xy}+y}  \\
& \\
& \\
\end{alignedat}
\end{equation}
%%%%%%%%%%%%%%%%%%%%%%%%%%%%%%%%%%%%%%%%%%%%%%%%%%%%%%%%%%%%%%%%
\section{Question} 

$\begin{aligned}[t] 
\text{If \ } x^p y^q = (x+y)^{p+q} \text{\ , \ the show that \ \  } \frac{dy}{dx}=\frac{y}{x}
\end{aligned}$

%----------------------------------------
\subsection*{Answer}
\begin{equation} \nonumber
\begin{alignedat}{4}
& \text{Given \ } x^py^q = (x+y)^{p+q} \\
& \therefore \log (x^py^q) = \log (x+y)^{p+q} \\
& \therefore p\log x + q\log y) = (p+q)\log (x+y) \\
& \text{Differentiating w.r.t. x \ }\\
& \therefore \frac{p}{x}+\frac{q}{y}\frac{dy}{dx} = \frac{p+q}{x+y}\left( 1+\frac{dy}{dx}\right)\\
& \therefore \frac{p}{x}+\frac{q}{y}\frac{dy}{dx} = \frac{p+q}{x+y} + \frac{p+q}{x+y}\,\frac{dy}{dx}\\
& \therefore \left( \frac{q}{y}-\frac{p+q}{x+y} \right)  \frac{dy}{dx} = \frac{p+q}{x+y} - \frac{p}{x}\\
\end{alignedat}
\,
\vrule
\, 
\begin{alignedat}{4}
& \therefore \frac{\cancel{qx - py}}{y} \,  \frac{dy}{dx} =  \frac{\cancel{qx - py}}{x}\\
& \therefore \frac{dy}{dx}=\frac{y}{x}\\
& \\
& \\
& \\
& \\
& \\
& \\
\end{alignedat}
\end{equation}

\begin{equation} \nonumber
\begin{alignedat}{4}
& \therefore \frac{q(x+y) - (p+q)y}{y(x+y)} \,  \frac{dy}{dx} = \frac{(p+q)x - p(x+y)}{(x+y)x}\\
& \therefore \frac{qx + qy - py - qy}{y} \,  \frac{dy}{dx} = \frac{px + qx - px - py}{x}\\
\end{alignedat}
\end{equation}


%%%%%%%%%%%%%%%%%%%%%%%%%%%%%%%%%%%%%%%%%%%%%%%%%%%%%%%%%%%%%%%%

\end{fleqn}
\end{document} 