\documentclass[17pt]{extarticle}
\usepackage{polyglossia}
\usepackage[a4paper, total={8in, 11.38in},top=2mm,left=27mm,bottom=2mm,right=1cm]{geometry}
\setmainlanguage{sanskrit}
\setmainfont{Gargi}

\begin{document}
बुद्धिबळ आणि मुक्त प्रणाली - १

सुमारे 25 वर्षां पूर्वी संगणक सामान्य लोकांसाठी उपलब्ध झाले. बुद्धिबळ खेळणारी सॉफ्टवेअर त्या बरोबरच आली.  चेस मास्टर आणि फ्रीडझ ही दोन प्रमुख व्यावसायिक उत्पादने तेव्हा प्रचंड लोकप्रिय होती. फ्रिडझ हे अधिक चांगल्या दर्जाचे होते. जागतिक दर्जाचे खेळाडू ते स्पर्धेत टिकून राहण्यासाठी वापरत असत.

जशी अधिकाधिक चेस सॉफ्टवेअर बाजारात आली तशी त्यांच्यामधे सुद्धा स्पर्धा घेण्यात येऊ लागल्या.  संगणकाला खरे तर पट आणि सोंगट्या लागत नाहीत. बुद्धिबळाच्या डावाची स्थिती सांकेतिक भाषेत मेमरी मधे असते. दिलेली चाल संगणकाला बुद्धिबळाच्या भाषेत दिली तर संगणक त्याच भाषेत पुढील चालीने उत्तर देतो. उदा.  1 e4 e5. 2. Nf3 Nc6.  ही भाषा कोणालाही केवळ पाच दहा मिनिटात शिकता येते.

 बुद्धिबळाच्या सॉफ्टवेअर्सचे सामान्यतः दोन भाग केलेले असतात. बुद्धिबळ खेळणारे इंजिन आणि इंजिनाने दिलेले उत्तर डिस्प्ले करणारी युझर इंटरफेस.  

बुद्धिबळ सॉफ्टवेअरची स्पर्धा म्हणजे बुद्धिबळ इंजिनाची स्पर्धा असते.

सुरवातीला फक्त व्यावसायिक चेस सॉफ्टवेअर्स उपलब्ध होती. ती अतिशय महाग होती. त्याची किंमत सामान्य लोकांना परवडणारी नव्हती. विद्यार्थ्यांना बुद्धिबळ शिकण्यासाठी स्वस्त आणि चांगल्या दर्जाच्या सॉफ्टवेअरची गरज होती. 

 मुक्त प्रणाली चळवळ सुरू झाली आणि GNU चेस हे मुक्त प्रणालीवर आधारित चेस इंजिन बुद्धिबळ प्रेमींना सर्वप्रथम खुले झाले. गेल्या पंचवीस वर्षात अनेक मुक्त बुद्धिबळ सॉफ्टवेअर आली आणि अतिशय प्रगत झाली. आज आनंद , कार्लसन , कास्पाराव यांच्या दर्जाची मुक्त सॉफ्टवेअर सहज उपलब्ध असताना शाळा ती विद्यार्थ्यांना वापरायला प्रोत्साहन देत नाहीत. किंबहुना त्याबद्दल खेळ आणि क्रीडाविभागाला माहितीच नाहीत असे वाटते.
\end{document}