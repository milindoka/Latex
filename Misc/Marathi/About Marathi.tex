\documentclass[17pt]{extarticle}
\usepackage{polyglossia}
\usepackage[a4paper, total={8in, 11.38in},top=2mm,left=27mm,bottom=2mm,right=1cm]{geometry}
\setmainlanguage{sanskrit}
\setmainfont{Gargi}

\begin{document}

मुखपृष्ठ
(शेवटचा बदल २९.१०.२०१८)
लाटेक् काय आहे?
लाटेक् (LaTeX) ही प्रबंध, अहवाल, लेख, पुस्तके, ग्रंथ, पत्रके,सादरीकरणे  (presentations), पत्रे, इ-पुस्तके, दिनदर्शिका, प्रशस्तिपत्रके असे आणि इतरही नानाविध दस्तऐवज संगणकावर निर्माण करण्यासाठी वापरली जाणारी संपादक आज्ञावली आहे. लाटेक् मुक्त आणि विनामूल्य उपलब्ध आहे. इतर कोणत्याही संपादकाहून उच्च प्रतीचे दस्तऐवज लाटेक् निर्माण करू शकते.   लाटेक् सतत आद्ययावत होत असते नि हे बदल आपल्याला विनामूल्य मिळतात. लाटेक् इ-पुस्तकेही लीलया बनवू शकते. १९८०च्या दरम्यान,  मुळात गणित संशोधकांकरता, बनवलेला हा संपादक आता संशोधनाच्या जवळपास सर्वच क्षेत्रांत वापरला जातो. जगभरातील संशोधनक्षेत्रातील ग्रंथप्रकाशक आताशा लाटेक् वापरत नाहीत, असे क्वचित् पहावयास मिळते. पाश्चिमात्य सर्वसाधारण पुस्तक-निर्मातेही आता लाटेक् मोठ्या प्रमाणावर वापरू लागले आहेत. लाटेक्-चे संगणकावरील आज्ञावलींसोबत मेतकूट फारच सहज नि उत्तमरीत्या जमते, हे लक्षात आल्याने आता पाश्चिमात्य इ-पुस्तक निर्माते, वेब-डिझाईनर आणि बॅन्काही त्याचा वापर करतात.

या प्रकल्पाबद्दल

आयसर, पुणे येथे २०१७च्या ऑक्टोबर महिन्यात लाटेक्-ची कार्यशाळा झाली. त्या दरम्यान पुणे, मुंबई, अहमदनगर, नाशिक अशा विविध ठिकाणांहून आलेल्या विविध लोकांनी मिळून महाराष्ट्राकरिता लाटेक्-गट निर्माण केला; हा एक स्वतंत्र गट असून तो लाटेक्-च्या TUGसह संलग्न नाही. या गटाच्या ३ फेब्रुवारी २०१८ रोजी झालेल्या बैठकीमधे काही मराठी ग्रंथ, पुस्तके वा चोपड्या लाटेक्-मध्ये बनवण्याची कल्पना मांडण्यात आली होती. ही बैठक भास्कराचार्य प्रतिष्ठान, पुणे येथे पार पडली. हा मुद्दा धरून, आम्ही ज्ञानेश्वरी, दासबोध, संपूर्ण बाळकराम, शिवलीलामृत आणि अजून काही ग्रंथ व चोपड्या लाटेक्-मधे बनवणे सुरू केले असून, हे काम आता अंतिम टप्प्यात आहे. आम्ही केलेले हे दस्तऐवज पीडीएफ् आणि इपब् अशा स्वरूपात बनवून येथे विनामूल्य उपलब्ध करून देत आहोत.

येथील सर्व मूळ रचना प्रताधिकारमुक्त आहेत. मूळ मजकूर विकीस्रोतावरून घेतला आहे वा गटाच्या सदस्यांनी टंकित केला आहे. हे दस्तऐवज बनवताना आम्ही पूर्णतः मुक्त आणि मोफत स्वरूपाची साधने वापरली आहेत. आम्ही वापरलेले टंकही मुक्त आणि मोफत स्वरूपाचे आहेत. मुक्त आणि मोफत टंक व मुक्त आणि मोफत साधने वापरून व्यवसायिक दर्जाचे दस्तऐवज निर्माण करता येतात, याची प्रचीती तुम्हाला हे दस्तऐवज पाहून येईल. आम्ही "एक टाईप" या गटास त्यांच्या सुंदर टंकाबद्दल धन्यवाद देऊ इच्छितो. लाटेक्, इमॅक्स्, गिम्प आणि इंकस्केप ही मुक्त आणि मोफक साधने आम्ही वापरली आहेत.

रोहित दिलीप होळकर, सुशान्त देवळेकर आणि प्राजक्ता रोहित होळकर यांनी या कामात महत्त्वाचे योगदान दिले आहे.

प्रताधिकार
येथे उपलब्ध असणाऱ्या दस्तऐवजांच्या संपादकीय रचनेकरिता आम्ही क्रिएटिव्ह कॉमन्सचे प्रताधिकार घेतले आहेत. शक्यतो हे प्रताधिकार मोफत, शेअर-अलाईक आणि व्यवसायिक स्वरूपाचे आहेत. मात्र काही दस्तऐवज अ-व्यवसायिक स्वरूपाचेही आहेत. प्रताधिकाराचे तपशील प्रत्येक दस्तऐवजावर दिले आहेत. या प्रताधिकारांचा वापरकर्त्यांनी आदर करावा. प्रताधिकारांचा भंग झाल्यास क्रिएटिव्ह कॉमन्सच्या वतीने कायदेशीर कार्यवाही केली जाईल.

\end{document}