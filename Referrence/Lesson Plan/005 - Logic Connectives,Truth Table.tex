\documentclass[17pt]{article}
\usepackage{amsmath}
\usepackage{booktabs}
\usepackage{amsmath, amssymb}
\usepackage{nccmath}
\usepackage[a4paper, total={8in, 11.38in},top=5mm,left=27mm,bottom=2mm,right=2mm]{geometry}


\newcommand{\lto}{\mathbin{\to}}

\begin{document}
\section{Logical Connectives}
Now something related to Logical connectives.

$\sim p$, $p \vee q$, 
$p \wedge q$, $p\to q$, 
$p \leftrightarrow q$. \\
Now let us combine the above operations the see the output !!! \\
1. $p\vee \sim q$ \\
2. $q \to ( r \to \sim q )$ \\
3. $p \leftrightarrow ( q \wedge  \sim r) $ \\
4. $(p \vee \sim q) \leftrightarrow [(p \wedge \sim q) \to (q \to \sim r)] $ \\
5. $(p \vee \sim q) \equiv [(p \wedge \sim q) \to (q \to \sim r)] $


\section{Question}
Draw the switching circuit for the following statement pattern \\ and also construct switching table : \ \  
$(p \wedge \sim q \wedge r ) \vee [ p \wedge ( \sim q \vee \sim r )] $



\section{Truth Table}
\begin{tabular}{|c*{5}{c|}}
\multicolumn{3}{c}{}&
\multicolumn{1}{c}{\textbf{A}}
&\multicolumn{1}{c}{}&
\multicolumn{1}{c}{\textbf{B}} \\
\hline
$p$ & $q$ & $p\land q$ & $(p\land q)\lto p$ & $p\lor q$ & $p\to (p\lor q)$\\
\hline
 T & T & T & T & T & T\\
 T & F & F & T & T & T\\
 F & T & F & T & T & T\\
 F & F & F & T & F & T\\
\hline
\end{tabular}

\end{document}
