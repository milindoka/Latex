\documentclass[17pt]{extarticle}
\usepackage{amsmath, amssymb}
\usepackage{nccmath}
\usepackage[a4paper, total={6.5in, 10.5in},top=5mm,left=27mm]{geometry}
\usepackage{titlesec}
\usepackage{tikz}
\usepackage{gensymb}
\usetikzlibrary{arrows.meta}

\titleformat{\section}
{\normalfont\normalsize\bfseries}{\thesection}{1em}{}
\setcounter{secnumdepth}{0} %% no numbering for sections

\newcommand{\lto}{\mathbin{\to}}

\begin{document}
\begin{fleqn}

%%%%%%%%%%%%%%%%%%%%%%%%%%%%%%%%%%%%%%%%%%%%%%%%%%%%%%%%%%%%%%%%%%%%%%%%%

\section{Question}
Show that 
$(p \wedge q)\to p$ $ \text{and} \ $p$ \to(p \vee q)$  \text{are equal} \\

%----------------------------------------------------------------------
\section{Solution} 
The truth table for the given statement pattern is as follows \\
\begin{tabular}{|c|*{5}{c|}}
  \multicolumn{3}{c}{}
& \multicolumn{1}{c}{\text{A}}
& \multicolumn{1}{c}{}
& \multicolumn{1}{c}{\text{B}} \\
\hline
$p$ & $q$ & $p\land q$ & $(p\land q)\lto p$ & $p\lor q$ & $p\to (p\lor q)$\\
\hline
 T & T & T & T & T & T \\ \hline
 T & F & F & T & T & T \\ \hline
 F & T & F & T & T & T \\ \hline
 F & F & F & T & F & T \\ \hline
\end{tabular} \\ \\
All the entries of column(A) and column(B) are identical \\
$\therefore$ \ $(p \wedge q)\to p$ $ \equiv \ $p$ \to(p \vee q)$ 

%%%%%%%%%%%%%%%%%%%%%%%%%%%%%%%%%%%%%%%%%%%%%%%%%%%%%%%%%%%%%%%%%%%%%%%%%%

\section{Question}
Using truth table show that $(p \wedge q)$ $ \equiv \ \sim(p \to \sim q)$ 

%----------------------------------------

\section{Solution}
The truth table for the given statement pattern is as follows \\
\begin{tabular}{|c|*{5}{c|}}
  \multicolumn{2}{c}{}
& \multicolumn{1}{c}{\text{A}}
& \multicolumn{1}{c}{}
& \multicolumn{1}{c}{}
& \multicolumn{1}{c}{\text{B}} \\
\hline
$p$ & $q$ & $p\land q$ & $ \sim q$ & $p\to \sim q$ & $\sim(p \to \sim q)$\\
\hline
 T & T & T & F & F & T \\ \hline
 T & F & F & T & T & F \\ \hline
 F & T & F & F & T & F \\ \hline
 F & F & F & T & T & F \\ \hline
\end{tabular} \\ \\
All the entries of column(A) and column(B) are identical \\
$\therefore$ \ $(p \wedge q) $ $ \equiv \ \sim(p \to \sim q)$ 

%%%%%%%%%%%%%%%%%%%%%%%%%%%%%%%%%%%%%%%%%%%%%%%%%%%%%%%%%%%%%%%%%%%%%

\section{Question}
Using truth table show that $(p \wedge q) \to r$ $ \equiv \ p \to (q \to r)$ 

%----------------------------------------

\section{Solution}
The truth table for the given statement pattern is as follows \\
\begin{tabular}{|c|*{7}{c|}}
  \multicolumn{4}{c}{}
& \multicolumn{1}{c}{\text{A}}
& \multicolumn{1}{c}{}
& \multicolumn{1}{c}{\text{B}}\\
\hline
$p$ & $q$ & $r$ & $p\land q$ & $ (p \wedge q) \to r$ & $q\to r$ & $p \to (q \to r)$\\
\hline
 T & T & T & T & T & T & T \\ \hline
 T & T & F & T & F & F & F \\ \hline
 T & F & T & F & T & T & T \\ \hline
 T & F & F & F & T & T & T \\ \hline
 F & T & T & F & T & T & T \\ \hline
 F & T & F & F & T & F & T \\ \hline
 F & F & T & F & T & T & T \\ \hline
 F & F & F & F & T & T & T \\ \hline

\end{tabular} \\ \\
All the entries of column(A) and column(B) are identical \\
$\therefore$ \ $(p \wedge q) \to r$ $ \equiv \ p \to (q \to r)$ 

%%%%%%%%%%%%%%%%%%%%%%%%%%%%%%%%%%%%%%%%%%%%%%%%%%%%%%%%%%%%%%%%%%%%%

\begin{equation} \nonumber
\end{equation}




%%%%%%%%%%%%%%%%%%%%%%%%%%%%%%%%%%%%%%%%%%%%%%%%%%%%%%%%%%%%%%%%%%%%%%%%%%

\end{fleqn}
\end{document} 