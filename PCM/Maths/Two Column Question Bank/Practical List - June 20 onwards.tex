\documentclass[14pt]{article}
\usepackage[a4paper, total={7.5in, 10.5in}]{geometry}
\usepackage[utf8]{inputenc}
\usepackage[english]{babel}
\usepackage{paralist} 
\usepackage{multicol}
\usepackage{amsmath}
\usepackage{amssymb}
\usepackage{enumitem}
\usepackage{textcomp}



\newcommand{\degree}{$^{\circ}\ $} %added space after degree

\setlength{\columnsep}{0.5cm}

\begin{document}
\centering 
{\large \bf XI Practicals\par}
\vspace{1cm}
\begin{multicols}{2}

%%%////////////////////////////////////
\section{Angle and its Measurement}
\noindent
\begin{enumerate}

\item Convert the following degree measures
in the radian measures : i) 70\degree ii) 120\degree iii) $\left(\frac{1}{4}\right)$\degree

\item The measures of the angles of the
triangle are in A. P. The smallest angle
is 40. Find the angles of the triangle in
degree and in radians.

\end{enumerate} 

%%%////////////////////////////////////
\section{Trigonometry - I}
\noindent
\begin{enumerate}
\item If $\tan \theta + \frac{1}{\tan \theta}=2$ then find the value of $\tan^2\theta + \frac{1}{\tan^2\theta}$ 

\item  Prove that $(\sec A - \tan A)^2=\frac{1-\sin A}{1+\sin A}$
\end{enumerate}

%%%////////////////////////////////////
\section{Trigonometry - II}
\noindent
\begin{enumerate}
\item Find the value of cos 15\degree

\item Prove that tan 20\degree tan 40\degree tan 60\degree tan 80\degree
= 3 

\end{enumerate} 

%%%////////////////////////////////////
\section{ Determinants and Matrices}
\noindent
\begin{enumerate}
\item Solve $
x+y+z = 6,  x-y+z = 2,  x+2y–z = 2$ using Cramer's Rule.

\item Solve $x +y - z = 1,  8x +3y - 6z = 1, -4x - y + 3z = 1$ using Cramer's Rule.
\end{enumerate} 

%%%////////////////////////////////////
\section{Straight Line}
\noindent
\begin{enumerate}
\item Find equations of lines which pass
through the origin and make an angle of 45\degree
with the line $3x - y = 6.$

\item Find the equation of line which passes
	 through the point of intersection of lines
	 $3 x + 2 y - 6 = 0 ,\; x + y + 1 = 0$ and the point A(2,1).

\end{enumerate} 

%%%////////////////////////////////////
\section{Circle}
\noindent
\begin{enumerate}

\item Find the equation of a circle whose centre is
(−3, 1) and which pass through the point (5, 2).

\item Find the equation of the tangent to the circle $x^2 + y^2 - 4x - 6y - 12 = 0$ at (-1, -1)

\end{enumerate} 

%%%////////////////////////////////////
\section{Conic Sections}
\noindent
\begin{enumerate}
\item Find the equation of tangent to the parabola
$y^2 = 9x$ at (1,-3).

\item Find the eccentricity of an ellipse whose
length of the latus rectum is one third of its
minor axis.

\end{enumerate} 

%%%////////////////////////////////////
\section{Measures of Dispersion}
\noindent
\begin{enumerate}
\item Given below are the marks out of 25
of 5 students in mathematics test. Calculate
the variance and standard deviation of these
observations. Marks : 10, 13, 17, 20, 23

A die is rolled 30 times and the following
distribution is obtained. Find the variance and
S.D.

\begin{tabular}{|c|*{11}{c|}}
\hline Score (X) & 1 & 2 & 3 & 4 & 5 & 6 \\
\hline Frequency (f) & 2 & 6 & 2 & 5 & 10 & 5 \\
\hline
\end{tabular}

\item

\end{enumerate} 

%%%////////////////////////////////////
\section{Probability}
\noindent
\begin{enumerate}
\item If P(A') = 0.7, P(B) = 0.7, P(B/A) =
0.5, find P(A/B) and P(A$\cup$B).

\item An urn contains 4 black and 6 white
balls. Two balls are drawn one after the other
without replacement, what is the probability that
both balls are black?

\end{enumerate} 

%%%////////////////////////////////////
\section{Complex Numbers}
\noindent
\begin{enumerate}
\item f a and b are real and
$(i^4 +3i)a + (i-1)b + 5i^3 = 0$ , find a and b.

\item Find the square root of $3 - 4i$.

\end{enumerate} 

%%%////////////////////////////////////
\section{Sequences and Series}
\noindent
\begin{enumerate}
\item Find three numbers in G.P. such that their
sum is 42 and their product is 1728.

\item For a G.P. if $S_3 = 16$, $S_6 =144$, find  the first
term and the common ratio of the G.P.



\end{enumerate} 

%%%////////////////////////////////////
\section{Permutations and Combination}
\noindent
\begin{enumerate}
\item How many different ways are there to
arrange letters of the word 'WORLD'? How many
of these arrangements begin with the letter R?
How many arrangements can be made taking
three letters at a time?

\item Find n and r
	if $^nC_{r-1}$ : $^nC_r$ : $^nC_{r+1} = 14:8:3$

\end{enumerate} 


%%%////////////////////////////////////
\section{Methods of Induction and Binomial Theorem}
\noindent
\begin{enumerate}
\item By method of induction, prove that.
	$5^{2n} - 1$ is divisible by 6, for all $n \in N$
	
\item Find the middle terms in the expansion of $\left(2x - \frac{1}{4x}\right)^9$

\end{enumerate} 

%%%////////////////////////////////////
\section{Sets and Relations}
\noindent
\begin{enumerate}
\item In a survey of 100 consumers 72
like product A and 45 like product B. Find the
least and the most number that must have liked
both products A and B.

\item A and B are two sets given in such a way
that A$\times$B contains 6 elements. If three elements of
A$\times$B are (1, 3), (2, 5) and (3, 3), find its remaining
elements.

\end{enumerate} 

%%%////////////////////////////////////
\section{Functions}
\noindent
\begin{enumerate}
\item Prove that, $2\log_ba^4 \cdot \log_cb^3 \cdot \log_ac^5 = 120$

\item If $f (x) = x^2$ , $g (x) = x + 5$, and $h(x) = \frac{1}{x}$ ,
	$x \neq 0$, find $(g \circ f \circ h) (x)$
\end{enumerate} 

%%%////////////////////////////////////
\section{Limits}
\noindent
\begin{enumerate}

\item Prove that $ \lim\limits_{x \to a} \frac{x^n - a^n}{x - a}=na^{n-1}$ where $n \in N,\;a>0.$

\item Evaluate : $ \lim\limits_{x \to 0} \frac{\tan\;x - sin\;x}{x^3} $ 

\end{enumerate} 

%%%////////////////////////////////////
\section{Continuity}
\noindent
\begin{enumerate}
\item If
$f(x) =  \begin{cases} 
       \frac{xe^x+\tan\;x}{\sin\;3x} & x\neq 0 \\
        k                            &  x = 0 
   \end{cases}$\, is continuous at $x=0$, find k.


\item Examine continuity of
$f(x) =  \begin{cases} 
       \frac{\log\;x - \log\;5}{x - 5} & x\neq 5 \\ 
        \frac{1}{5}                           &  x = 5
   \end{cases}$ at $ x  = 5$


\end{enumerate} 

%%%////////////////////////////////////
\section{Differentiation}
\noindent
\begin{enumerate}
\item Find the derivative of $\sqrt{x}$ from the definition.

\item If $f(x) = p \tan x + q \sin x + r, f(0) = -4\sqrt{3}$ , $f\left(\frac{\pi}{3}\right)=-7\sqrt{3} $, $f'\left(\frac{\pi}{3}\right)=3$, then find $ p,q,r.$
\end{enumerate} 


\end{multicols}
 
\end{document}