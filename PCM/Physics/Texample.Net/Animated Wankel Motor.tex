% Animated Wankel-Motor
% Author: Robert Papanicola
\documentclass{article} 
\usepackage{tikz}
%%%<
\usepackage{verbatim}
\usepackage[active,tightpage]{preview}
\setlength\PreviewBorder{5pt}%
%%%>
\begin{comment}
:Title: Animated Wankel-Motor
:Tags: Animations;Physics
:Author: Robert Papanicola
:Slug: wankel-motor

From Wikipedia: The Wankel engine is a type of internal combustion engine
using an eccentric rotary design to convert pressure into a rotating motion
instead of using reciprocating pistons. Its four-stroke cycle takes place in
a space between the inside of an oval-like epitrochoid-shaped housing and
a rotor that is similar in shape to a Reuleaux triangle but with sides that
are somewhat flatter. The very compact Wankel engine delivers smooth
high-rpm power. It is commonly called a rotary engine, though this name
applies also to other completely different designs. It is the only internal
combustion engine invented in the twentieth century to go into production.
\end{comment}%\usepackage{pgfmath}
\usetikzlibrary{calc}
\usepackage{animate}
\usepackage{fp} %[fr] utile pour les calculs de position des différentes phases du moteur
% [fr] mais inutile pour la simulation seule
% Useful for calculating the position of the different phases of the motor but useless for simulation only

\newcommand{\Wankel}[1]{
% \def\itheta{#1}
\FPabs{\itheta}{#1}
% définition des données \ data definition
\def\OA{0.4}
\def\OB{0.8}
\def\AE{4}
% \def\seuil{500}
% \def\couleur{20}

%==========
% [fr] définition des paramètres angulaires du rotor en fonction des phases du moteur 
% definition of angular parameters of the rotor according to the motor phases
\def\Comp{0}
\def\Expl{360}
\def\Det{375}
\def\Ech{660}
\def\Asp{870}
\def\decalage{125} 
% [fr] décalage de l'origine pour positionner le rotor au début de la compression à l'instant t=0
% [en] shift the origin to position the rotor at the start of compression at time t = 0
\begin{tikzpicture}
%===== [fr] Définition de quelques couleurs pales\ [en]  some color
\colorlet{vertclair}{green!25}
\colorlet{grisclair}{gray!60}
\colorlet{rougepale}{red!60}

% [fr] Début des test nécessaire pour colorer la chambre
% test needed to color the combustion chamber
\FPabs{\val}{\itheta} 
% [fr] FPiflt et les autres tests de FP ne permettent pas de faire des
% [fr] tests inclus dans des tests. Le premier test est donc toujours vrai,
% [fr] le style chambre1 sera donc affecté avec [{vertclair!\pos!orange}]
% [fr] mais ne sera affiché que si aucun des tests suivants n'est vrai.% Animated Wankel-Motor
% Author: Robert Papanicola
\documentclass{article} 
\usepackage{tikz}
%%%<
\usepackage{verbatim}
\usepackage[active,tightpage]{preview}
\setlength\PreviewBorder{5pt}%
%%%>
\begin{comment}
:Title: Animated Wankel-Motor
:Tags: Animations;Physics
:Author: Robert Papanicola
:Slug: wankel-motor

From Wikipedia: The Wankel engine is a type of internal combustion engine
using an eccentric rotary design to convert pressure into a rotating motion
instead of using reciprocating pistons. Its four-stroke cycle takes place in
a space between the inside of an oval-like epitrochoid-shaped housing and
a rotor that is similar in shape to a Reuleaux triangle but with sides that
are somewhat flatter. The very compact Wankel engine delivers smooth
high-rpm power. It is commonly called a rotary engine, though this name
applies also to other completely different designs. It is the only internal
combustion engine invented in the twentieth century to go into production.
\end{comment}%\usepackage{pgfmath}
\usetikzlibrary{calc}
\usepackage{animate}
\usepackage{fp} %[fr] utile pour les calculs de position des différentes phases du moteur
% [fr] mais inutile pour la simulation seule
% Useful for calculating the position of the different phases of the motor but useless for simulation only

\newcommand{\Wankel}[1]{
% \def\itheta{#1}
\FPabs{\itheta}{#1}
% définition des données \ data definition
\def\OA{0.4}
\def\OB{0.8}
\def\AE{4}
% \def\seuil{500}
% \def\couleur{20}

%==========
% [fr] définition des paramètres angulaires du rotor en fonction des phases du moteur 
% definition of angular parameters of the rotor according to the motor phases
\def\Comp{0}
\def\Expl{360}
\def\Det{375}
\def\Ech{660}
\def\Asp{870}
\def\decalage{125} 
% [fr] décalage de l'origine pour positionner le rotor au début de la compression à l'instant t=0
% [en] shift the origin to position the rotor at the start of compression at time t = 0
\begin{tikzpicture}
%===== [fr] Définition de quelques couleurs pales\ [en]  some color
\colorlet{vertclair}{green!25}
\colorlet{grisclair}{gray!60}
\colorlet{rougepale}{red!60}

% [fr] Début des test nécessaire pour colorer la chambre
% test needed to color the combustion chamber
\FPabs{\val}{\itheta} 
% [fr] FPiflt et les autres tests de FP ne permettent pas de faire des
% [fr] tests inclus dans des tests. Le premier test est donc toujours vrai,
% [fr] le style chambre1 sera donc affecté avec [{vertclair!\pos!orange}]
% [fr] mais ne sera affiché que si aucun des tests suivants n'est vrai.
% FPiflt and other FP tests do not do tests included in the tests. The first
% test is always true, the style will be affected with chambre1
% [{vertclair! \ Pos! Orange}] but will not be displayed if any of the
% following tests is true.
\FPiflt{\val}{1080}
\FPeval{\pos}{(\val-1080)/(\Ech-1080)*100} 
\tikzstyle{chambre1}=[{vertclair!\pos!orange}]% aspiration
% \tikzstyle{chambre2}=[ ball color={gray!\pos!red}]% détente}
% \tikzstyle{chambre3}=[ ball color={orange!\pos!green}]% aspiration
\def\texte{Aspiration}
\fi

% [fr] echappement / Second test, Exhaust
\FPiflt{\val}{\Asp}
\FPeval{\pos}{(\val-\Ech)/(\Asp-\Ech)*100} 
\tikzstyle{chambre1}=[{vertclair!\pos!grisclair}]%echappement
% \tikzstyle{chambre2}=[ ball color={gray!\pos!red}]% détente}
% \tikzstyle{chambre3}=[ ball color={orange!\pos!green}]%aspiration
\def\texte{Echappement}
\fi

% [fr] troisième test - détente  /  Third test - relaxation

\FPiflt{\val}{\Ech}
\FPeval{\pos}{(\val-\Det)/(\Ech-\Det)*100} 
\tikzstyle{chambre1}=[ {grisclair!\pos!rougepale}]%détente}
% \tikzstyle{chambre2}=[ ball color={gray!\pos!red}]%compression
% \tikzstyle{chambre3}=[ ball color={orange!\pos!green}]%aspiration
\def\texte{Detente}
\fi

% [fr] quatrième test - explosion /  Fourth test - explosion
\FPiflt{\val}{\Det}
\FPeval{\pos}{(\val-\Expl)/(\Det-\Expl)*100} 
\tikzstyle{chambre1}=[ {rougepale!\pos!red}]%Explosion
% \tikzstyle{chambre2}=[ ball color={gray!\pos!red}]%détente}
% \tikzstyle{chambre3}=[ ball color={orange!\pos!green}]%aspiration
\def\texte{Explosion}
\fi

% [fr] cinquième test / Fifth test - aspiration
\FPiflt{\val}{\Expl}
\FPeval{\pos}{(\val-\Comp)/(\Expl-\Comp)*100} 
\tikzstyle{chambre1}=[ {red!\pos!orange}]%compression
% \tikzstyle{chambre2}=[ ball color={gray!\pos!red}]%détente}
% \tikzstyle{chambre3}=[ ball color={orange!\pos!green}]%aspiration
\def\texte{Compression}
\fi

\FPtrunc{\pos}{\pos}{0}

% [fr] Ajout du décalage pour dessiner le rotor en position initiale
% [en] Adding the offset to draw the rotor in the first position
\FPeval{\itheta}{0-(\decalage+\itheta)}

% [fr] début du dessin 
\draw (-5,-4.5) rectangle (5,4.5); %cadre pour imposer les dimensions du dessin

%[fr}dessin du stator / drawing of the stator
\begin{scope}% stator
  \coordinate (A) at (\itheta:\OA);  %[fr] Le point A tourne autour de O
  % [fr] avec l'angle itheta
  % [en]Point A turns around point O the angle with itheta
  \coordinate (O) at (0,0);% Origine

  \filldraw[thick,black,domain=0:1080,smooth,variable=\t,fill=gray,samples=50]
  plot ({.4*cos(\t)+4*cos(.333333*\t)},{.4*sin(\t)+4*sin(.333333*\t)})
  plot ({.42*cos(-\t)+4.2*cos(-.333333*\t)},{.42*sin(-\t)+4.2*sin(-.333333*\t)});
  \fill[white](0.3,3) rectangle (0.5,4); % dessin des soupapes
      % (admission echappement) 
  \fill[white](-0.3,3) rectangle (-0.5,4);
  \filldraw[red](-0.4,-3.5) rectangle (-0.5,-3.7);% Bougie

  % [fr]Coloriage des chambres en fonction de la phase de fonctionnement,
  % [fr]le rotor sera dessine par dessus et masquera le surplus
  % [fr](le bord intérieur est linéaire et non incurvé).
  % Coloring of the rooms according to the operating phase, the rotor will be
  % plotted on top and hide the rest (the inner edge is straight, not curved).
  % Chambre 1
  \fill[thick,black,domain=-1*(\itheta):-1*(\itheta+360),smooth,variable=\t,
    chambre1] plot ({.4*cos(-\t)+4*cos(-.333333*\t)},
    {.4*sin(-\t)+4*sin(-.333333*\t)});

  % affichage des paramètres
  % \FPtrunc{\itheta}{\itheta}{0}
  % \FPtrunc{\val}{\val}{0}
  % \node at (0,6){$\theta$=\itheta,pos=\pos,val=\val};

  \draw[black](O) circle (\OB); %Dessin du pignon fixe
  \draw(O) -- (A);
\end{scope}

% [fr] Dessin du rotor / Draw the rotor
\begin{scope}[shift={(A)},rotate={\itheta}] % le repere est tourné de itheta
  % [fr] les trois points, C, D, E sont définis en polaire dans ce repère
  % [en] the three points, C, D, E are defined in the polar reference
  \coordinate (E) at ({-\itheta*\OB/(\OB+\OA)}:\AE); 
  \coordinate (C) at ({-\itheta*\OB/(\OB+\OA)+120}:\AE);
  \coordinate (D) at ({-\itheta*\OB/(\OB+\OA)+240}:\AE);
  \draw[red](A) -- (E);
  \filldraw [bend left=29.5,red,fill=red!50] (A) circle (\OA+\OB)% dessin et coloriage du rotor
    (E) to (D) to (C) to (E);
\end{scope}

\node at (0,-4) {\texte};  % le texte affiche la phase de fonctionnement
\end{tikzpicture}
}

\begin{document}
\begin{preview}
% [fr] Animation avec le package animate
% Animation with the animate package
\begin{animateinline}[controls,loop]{12}
  \multiframe{72}{ixb=0+15}{
    \Wankel{\ixb}}
\end{animateinline}
\vfill
%{fr] Pour dessiner le rotor dans un position particulière
% To draw the rotor in a particular position
\begin{center}
  \Wankel{200}
\end{center}
\end{preview}
\end{document}
% FPiflt and other FP tests do not do tests included in the tests. The first
% test is always true, the style will be affected with chambre1
% [{vertclair! \ Pos! Orange}] but will not be displayed if any of the
% following tests is true.
\FPiflt{\val}{1080}
\FPeval{\pos}{(\val-1080)/(\Ech-1080)*100} 
\tikzstyle{chambre1}=[{vertclair!\pos!orange}]% aspiration
% \tikzstyle{chambre2}=[ ball color={gray!\pos!red}]% détente}
% \tikzstyle{chambre3}=[ ball color={orange!\pos!green}]% aspiration
\def\texte{Aspiration}
\fi

% [fr] echappement / Second test, Exhaust
\FPiflt{\val}{\Asp}
\FPeval{\pos}{(\val-\Ech)/(\Asp-\Ech)*100} 
\tikzstyle{chambre1}=[{vertclair!\pos!grisclair}]%echappement
% \tikzstyle{chambre2}=[ ball color={gray!\pos!red}]% détente}
% \tikzstyle{chambre3}=[ ball color={orange!\pos!green}]%aspiration
\def\texte{Echappement}
\fi

% [fr] troisième test - détente  /  Third test - relaxation

\FPiflt{\val}{\Ech}
\FPeval{\pos}{(\val-\Det)/(\Ech-\Det)*100} 
\tikzstyle{chambre1}=[ {grisclair!\pos!rougepale}]%détente}
% \tikzstyle{chambre2}=[ ball color={gray!\pos!red}]%compression
% \tikzstyle{chambre3}=[ ball color={orange!\pos!green}]%aspiration
\def\texte{Detente}
\fi

% [fr] quatrième test - explosion /  Fourth test - explosion
\FPiflt{\val}{\Det}
\FPeval{\pos}{(\val-\Expl)/(\Det-\Expl)*100} 
\tikzstyle{chambre1}=[ {rougepale!\pos!red}]%Explosion
% \tikzstyle{chambre2}=[ ball color={gray!\pos!red}]%détente}
% \tikzstyle{chambre3}=[ ball color={orange!\pos!green}]%aspiration
\def\texte{Explosion}
\fi

% [fr] cinquième test / Fifth test - aspiration
\FPiflt{\val}{\Expl}
\FPeval{\pos}{(\val-\Comp)/(\Expl-\Comp)*100} 
\tikzstyle{chambre1}=[ {red!\pos!orange}]%compression
% \tikzstyle{chambre2}=[ ball color={gray!\pos!red}]%détente}
% \tikzstyle{chambre3}=[ ball color={orange!\pos!green}]%aspiration
\def\texte{Compression}
\fi

\FPtrunc{\pos}{\pos}{0}

% [fr] Ajout du décalage pour dessiner le rotor en position initiale
% [en] Adding the offset to draw the rotor in the first position
\FPeval{\itheta}{0-(\decalage+\itheta)}

% [fr] début du dessin 
\draw (-5,-4.5) rectangle (5,4.5); %cadre pour imposer les dimensions du dessin

%[fr}dessin du stator / drawing of the stator
\begin{scope}% stator
  \coordinate (A) at (\itheta:\OA);  %[fr] Le point A tourne autour de O
  % [fr] avec l'angle itheta
  % [en]Point A turns around point O the angle with itheta
  \coordinate (O) at (0,0);% Origine

  \filldraw[thick,black,domain=0:1080,smooth,variable=\t,fill=gray,samples=50]
  plot ({.4*cos(\t)+4*cos(.333333*\t)},{.4*sin(\t)+4*sin(.333333*\t)})
  plot ({.42*cos(-\t)+4.2*cos(-.333333*\t)},{.42*sin(-\t)+4.2*sin(-.333333*\t)});
  \fill[white](0.3,3) rectangle (0.5,4); % dessin des soupapes
      % (admission echappement) 
  \fill[white](-0.3,3) rectangle (-0.5,4);
  \filldraw[red](-0.4,-3.5) rectangle (-0.5,-3.7);% Bougie

  % [fr]Coloriage des chambres en fonction de la phase de fonctionnement,
  % [fr]le rotor sera dessine par dessus et masquera le surplus
  % [fr](le bord intérieur est linéaire et non incurvé).
  % Coloring of the rooms according to the operating phase, the rotor will be
  % plotted on top and hide the rest (the inner edge is straight, not curved).
  % Chambre 1
  \fill[thick,black,domain=-1*(\itheta):-1*(\itheta+360),smooth,variable=\t,
    chambre1] plot ({.4*cos(-\t)+4*cos(-.333333*\t)},
    {.4*sin(-\t)+4*sin(-.333333*\t)});

  % affichage des paramètres
  % \FPtrunc{\itheta}{\itheta}{0}
  % \FPtrunc{\val}{\val}{0}
  % \node at (0,6){$\theta$=\itheta,pos=\pos,val=\val};

  \draw[black](O) circle (\OB); %Dessin du pignon fixe
  \draw(O) -- (A);
\end{scope}

% [fr] Dessin du rotor / Draw the rotor
\begin{scope}[shift={(A)},rotate={\itheta}] % le repere est tourné de itheta
  % [fr] les trois points, C, D, E sont définis en polaire dans ce repère
  % [en] the three points, C, D, E are defined in the polar reference
  \coordinate (E) at ({-\itheta*\OB/(\OB+\OA)}:\AE); 
  \coordinate (C) at ({-\itheta*\OB/(\OB+\OA)+120}:\AE);
  \coordinate (D) at ({-\itheta*\OB/(\OB+\OA)+240}:\AE);
  \draw[red](A) -- (E);
  \filldraw [bend left=29.5,red,fill=red!50] (A) circle (\OA+\OB)% dessin et coloriage du rotor
    (E) to (D) to (C) to (E);
\end{scope}

\node at (0,-4) {\texte};  % le texte affiche la phase de fonctionnement
\end{tikzpicture}
}

\begin{document}
\begin{preview}
% [fr] Animation avec le package animate
% Animation with the animate package
\begin{animateinline}[controls,loop]{12}
  \multiframe{72}{ixb=0+15}{
    \Wankel{\ixb}}
\end{animateinline}
\vfill
%{fr] Pour dessiner le rotor dans un position particulière
% To draw the rotor in a particular position
\begin{center}
  \Wankel{200}
\end{center}
\end{preview}
\end{document}